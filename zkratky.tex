\newglossaryentry{DCC}{
	name=DCC,
	description={Digital Command Control, digitální systém řízení modelového
	kolejiště}
}
\newglossaryentry{BEMF}{
	name=BEMF,
	description={Back Electro-motive force, technologie pro měření rychlosti
	ovládaného motoru}
}
\newglossaryentry{WSM}{
	name=WSM,
	description={Obchodní název vyvinutého měřicího vozu: \textit{WSM:
	A~Wireless Speedometer}}
}
\newglossaryentry{CV}{
	name=CV,
	description={Configuration Value, jedna konfigurační jednotka s~rozsahem
	hodnot 1~byte ($0$--$255$). Každé CV má svoje číslo (typicky $1$--$1023$)}
}
\newglossaryentry{JOP}{
	name=JOP,
	description={Jednotné obslužné pracoviště, standard grafického rozhraní
	počítačového řízení skutečné železnice definovaný Správou železniční
	a~dopravní cesty}
}
\newglossaryentry{LDO}{
	name=LDO,
	description={Low-dropout regulator, lineární regulátor napětí s~velice nízkým
	dropout napětím (řádově malé desetiny voltu)}
}
\newglossaryentry{ICP}{
	name=ICP,
	description={Input Capture Pin, speciální pin procesoru ATmega328p určený
	k~tomu, aby přesně a~rychle měřil periodu vstupního signálu}
}
\newglossaryentry{BT}{
	name=BT,
	description={Bluetooth}
}
\newglossaryentry{SPP}{
	name=SPP,
	description={Serial Port Profile, profil komunikačního standardu Bluetooth}
}
\newglossaryentry{POM}{
	name=POM,
	description={Programming on Main, režim programování dekodérů DCC, kdy je
	dekodér programován na plného provozu na kolejišti bez nutnosti přepínání
	do programovacího režimu}
}
\newglossaryentry{LT}{
	name=LT,
	description={Long-term measure, režim měření rychlosti měřicím vozem,
	kdy je rychlost měřena po delší časový úsek}
}

\printglossary[title=Seznam použitých zkratek]
