\documentclass[aspectratio=169]{beamer}

\mode<presentation> {
	\usetheme{Madrid}
	\usecolortheme{beaver}
}

\beamertemplatenavigationsymbolsempty

\usepackage[czech]{babel}
\usepackage[utf8]{inputenc}
\usepackage[T1]{fontenc}
\usepackage{lmodern}
\usepackage{graphicx}
\usepackage{booktabs}
\usepackage{hyperref}

\deftranslation[to=Czech]{Example}{Příklad}

\title[Automatická kalibrace]{Systém automatické kalibrace modelového
železničního vozidla}

\author{Jan Horáček}
\institute[FI MUNI]{
	Fakulta informatiky \\
	Masarykova univerzita \\
	\medskip
	\textit{horacekj@mail.muni.cz}
}
\date{\today}

\begin{document}

%------------------------------------------------

\begin{frame}
\titlepage
\end{frame}

%------------------------------------------------

\begin{frame}
\frametitle{Cíl práce}
\begin{block}{Cíl práce}
Cílem práce je automatizovat proces kalibrace modelové lokomotivy.
\end{block}
\end{frame}

%------------------------------------------------

\begin{frame}
\frametitle{Kontext: řízení kolejiště}
TODO Obrázek terminál JOP.
\end{frame}

%------------------------------------------------

\begin{frame}
\frametitle{Kontext: digitální řízení vlaku}
Digital Command Control (DCC):
\begin{itemize}
\item Každá lokomotiva obsahuje miniaturní počítač, tzv. \uv{dekodér}.
\item DCC centrála generuje data pro lokomotivy.
\item Lokomotiva data přijímá a na základě nich koná akce.
\item Každá lokomotiva má svou číselnou adresu, kterou je identifikována (typicky
$1$--$9999$).
\item Centrála generuje DCC signál na základě povelů z~ovladačů.
\begin{itemize}
\item ruční ovladač, počítač, ...
\end{itemize}
\end{itemize}

TODO obrázek topologie kolejiště
\end{frame}

%------------------------------------------------

\end{document}
