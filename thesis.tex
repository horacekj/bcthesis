\documentclass[digital, twoside, notable, lot]{fithesis3}

%% The following section sets up the locales used in the thesis.
\usepackage[resetfonts]{cmap}
\usepackage[T1]{fontenc}
\usepackage[main=czech, english]{babel}

%% The following section sets up the metadata of the thesis.
\thesissetup{
    date          = \the\year/\the\month/\the\day,
    university    = mu,
    faculty       = fi,
    type          = bc,
    author        = Jan Horáček,
    gender        = m,
    advisor       = Zdeněk Matěj,
    title         = {Automatická kalibrace modelového hnacího vozidla},
    TeXtitle      = {Automatická kalibrace modelového hnacího vozidla},
    keywords      = {kalibrace, dcc, vlak, model, senzor},
    TeXkeywords   = {kalibrace, dcc, vlak, model, senzor},
    abstract      = {Tady bude abstrakt.},
    thanks        = {Děkuji Losu Karlosovi za skvělý metaprogram!},
    bib           = bibliography.bib,
}
\usepackage{makeidx}      %% The `makeidx` package contains
\makeindex                %% helper commands for index typesetting.
%% These additional packages are used within the document:
\usepackage{paralist} %% Compact list environments
\usepackage{amsmath}  %% Mathematics
\usepackage{amsthm}
\usepackage{amsfonts}
\usepackage{url}      %% Hyperlinks
\usepackage{listings} %% Source code highlighting
\lstset{
  basicstyle      = \ttfamily,%
  identifierstyle = \color{black},%
  keywordstyle    = \color{blue},%
  keywordstyle    = {[2]\color{cyan}},%
  keywordstyle    = {[3]\color{olive}},%
  stringstyle     = \color{teal},%
  commentstyle    = \itshape\color{magenta}}
\usepackage{floatrow} %% Putting captions above tables
\floatsetup[table]{capposition=top}
\begin{document}

\chapter*{Úvod}
\addcontentsline{toc}{chapter}{Úvod}

Říká se, že závěrečné práce jsou vyvrcholením studia a tak jsem se
rozhodl jednu také napsat. Pokud vše půjde podle plánu, odnesu si
na konci semestru diplom. Držte mi palce!

\chapter{Úvod} \label{chap:uvod}
V~následujících několika stranách autor rozebere možnosti řešení problému
automatické kalibrace modelového žlezničního hnacího vozidla, ukáže, proč
jsou současná řešení pro problém řešený v~této páci nevhodná a nakonec
navrhne (1) hardware, (2) firmware a (3) software, kterým se pokusí problém
řešit. Autor demonstruje výstupy této práce v~praxi a ukáže její reálný
přínos.

Tato práce se tématicky pohybuje na pomezí práce typu \textit{proof of concept}
a práce \textit{aplikační}. Jejím primárním cílem je vytvořit reálné řešení
reálného problému v~oblasti automatizace, konkrétně v~oblasti řízení dopravy na
modelovém kolejišti. Jakkoliv se může zdát, že toto téma se nachází velice
blízko hardwaru, primárním cílem této práce je navrhnout kvalitní softwarové
řešení.

Pro přesnou formulací problému automatické kalibrace železničního vozidla je
nutné pochopit kontext této práce, o kterém pojednává následující kapitola.

\section{Modelová kolejiště}

Modelové kolejiště je model reálné železniční tratě, typicky zmenšený
v~některém ze standardních měřítek, např. v~měřítku \uv{TT} (1:120). Vedle
typického \uv{hobby} využití kolejiště často slouží jako dopravní trenažéry pro
výuku zabezpečovacích zařízení železnice, či dalších aspektů využívaných na
skutečné železnici. Nejen v~těchto případech vzniká potřeba kolejiště ovládat
\textit{zabezpečovacím zařízením}. Typickým zástupcem současného
zabezpečovacího zařízení na skuteční železnici je \textit{elektronické
stavědlo} spolu s~grafickou nadstavbou \textit{JOP}, která umožňuje řízení
stavědla přes počítač.

Obdobný systém je využíván také v~\textit{Klubu modelářů železnic Brno I},
který je základní motivací k~vypracování této práce. V~tomto klubu je nasazen
systém řízení kolejišť zvaný \textit{hJOP} \cite{hjop:web}, který je klonem JOP
upraveným pro potřeby řízení modelu. Jeho primárním cílem je povelovat jízdu
vlaku a omezením lidského faktoru předejít možným kolizím.

\section{Současné trendy v~řízení kolejiště}

K~řízení modelových kolejišť je v~současné době napříč zeměmi využívána celá
řada dostupných SW. Typickými příklady jsou například \textit{JMRI}
\cite{jmri:web}, \textit{TrainController} \cite{trancontroller:web},
\textit{RailCo} \cite{railco:web} nebo český SW \textit{modelJOP}
\cite{modeljop:web}.

Výše zmíněné aplikace komunikují s~hardwarovými prvky kolejiště.
Nejdůležitějším z~těchto prvků je digitální centrála DCC. Digitální centrála je
obecný pojem, konkrétními instancemi od jednotlivých výrobců jsou pak např.
\textit{Z21} \cite{z21:web}, \textit{NanoX} \cite{nanox:web} nebo třeba
\textit{Intellibox} \cite{intellibox:web}. Centrála typicky komunikuje
s~počítačem pomocí sběrnice \textit{RS232 over USB}. Jejím hlavním úkolem je
vytvářet \textit{digitální signál DCC}, který moduluje do kolejí.

Lokomotiva stojící na kolejích tento signál dekóduje pomocí speciální
součástky -- tzv. \textit{dekodéru} -- a podle přijatých dat poveluje periferie
lokomotivy: motor, světla, nebo třeba zvuk vydávaný vestavěným reproduktorem.
Signál v~kolejích je zároveň využíván jako výkonový (např. pro napájení
motoru). Dekodér v~podstatě obsahuje jen jednoduchý mikrokontrolér a několik
výkonových prvků. Jeho typická velikost je v~řádu malých jednotek $cm^2$, takže
je skutečně malý.

Každý dekodér má svoji adresu v rozsahu $1-9999$. Počítač pak může vydat příkaz
pro řízení lokomotivy, např.: \textit{\uv{lokomotivo s~adresou 562, jeď směrem
vpřed rychlostí 15!}}

\section{Synchronizace rychlosti vozidel}

Rychlost lokomotivy typický bývá přirozené číslo v~rozsahu $0-28$. Tato hodnota
se označuje jako tzv. \textit{jízdní stupeň}.

Protože každá lokomotiva obsahuje jiné mechanické prvky (převody, motor),
odpovídá stejnému jízdnímu stupni u~dvou různých hnacích vozidel různá
rychlost. Ukazuje se, že při řízení kolejiště je vhodné spárovat jízdní stupeň
s~reálnou rychlostí vozidla. Je to především proto, abychom dosáhli modelové
věrnosti a vozidlo se po přepočtu pohybovalo \textit{modelovou rychlostí}. Je
také nutné dbát na provozní parametry kolejiště, které umožňují průjezd
některými části kolejiště jen omezenou rychlostí. Při překročení této rychlosti
by pak mohlo dojít k~vykolejení, což je samozřejmě nežádoucí.

Dalším důvodem pro spárování rychlosti lokomotivy s~jízdním stupněm je
vytvoření předvídatelného prostředí pro obsluhu. Obsluha má typicky k~dispozici
ovladač pro řízení rychlosti jízdy vozidla v~ručním režimu. Autor této práce
považuje za velice užitečné, aby platilo, že otočení trimru ovladače o~daný
úhel způsobí u~všech vozidel stejnou rychlost.

Třetím důvodem k synchronizaci rychlostí vozidel je umožnění přípřeží a postrků
-- tj.  situací, kdy v~jednom vlaku jede více lokomotiv. Tyto lokomotivy se
musí pohybovat stejnou rychlostí, jinak hrozí vykolejení vlaku.

\subsection{Současné řešení problému}

Současné řídicí programy problém synchronizace rychlostí typicky řeší tak, že
si u~každého hnacího vozidla ukládají tabulku \textit{jízdní stupeň: skutečná
rychlost}. Když tyto programy chtějí, aby 2 lokomotivy jely stejnou rychlostí,
provedou vyhledání příslušných jízdních stupňů a do centrály (v~obecném
případě) odešlou 2 různé jízdní stupně.

Provozu lokomotivy na kolejišti tak předchází buď

\begin{itemize}
	\item ruční zadání této tabulky, nebo
	\item automatické měření rychlosti lokomotivy.
\end{itemize}

Je však nutné podotknout, že toto řešení neuspokojuje naše požadavky pro
ruční řízení jízdy hnacího vozidla, protože tabulka rychlostí je uložena
pouze v~řídícím SW, s~kterým nejsou ovladače typicky propojeny, takže o~tabulce
vůbec netuší.

\subsection{Řešení problému v~hJOP}

Alternativním řešením problému synchronizace rychlostí je způsob, který
v~současné době využívá SW hJOP. Tato práce cílí právě na tento alternativní
způsob řešení kalibrace. Jeho výhoda je v~tom, že řeší synchronizaci
rychlostí i~pro ruční ovladače.

V~každém lokomotivním dekodéru je možno pro každý z~28 jízdních stupňů nastavit
konkrétní výkon motoru. SW hJOP staví na předpokladu, že uživatel provede před
provozem hnacího vozidla tzv. \textit{kalibraci}, tj. přiřazení výkonu motoru
každému rychlostnímu stupni každého vozidla tak, aby bylo splněno, že konkrétní
jízdní stupeň odpovídá u~všech vozidel konkrétní rychlosti.

Příklad: u~všech lokomotiv platí, že jízdní stupeň $15$ odpovídá reálné
rychlosti $40\ km/h$.

Dalšími výhodami tohoto řešení jsou:

\begin{enumerate}
	\item odpadnutí nutnosti udržovat si u~každého vozidla v~SW kalibrační
	tabulku
	\item a instantní přenos tabulky mezi kolejišti, když je přenesena
	lokomotiva. To proto, že tabulka je jednoduše fyzicky uložená v~lokomotivě a
	tudíž logicky cestuje s ní.
\end{enumerate}

Nevýhodou tohoto přístupu je, že je nutné provést netriviální proces kalibrace.
V~současné době tento proces zahrnuje ježdění s~lokomotivou na měřícím okruhu,
měření její rychlosti a ruční nastavování kalibrační tabulky.

Cílem této práce je tento proces automatizovat.

\section{Cíl práce}

Cílem této práce je automatizovat proces kalibrace lokomotivy.

Nejprve je nutné navrhnout metodu pro měření rychlosti hnacího vozidla. Jakmile
bude zabezpečen přenos dat o~rychlosti hnacího vozidla do řídícího PC, je nutné
navrhnout SW, který na nákladě změřené rychlosti provede nastavení kalibrační
tabulky hnacího vozidla.


\chapter{Požadavky na řešení} \label{chap:pozadavky}
Pro návrh vhodného řešení procesu automatické kalibrace je nutné formulovat
požadavky kladené na toto řešení. Základní požadavky seřazené od
nejdůležitějšího jsou:

\begin{compactenum}
	\item snadné použití pro koncového uživatele, snadná instalace,
	\item možnost použít řešení pro libovolné modelové měřítko,
	\item maximalizace automatizace a minimalizace nutné spolupráce s~uživatelem,
	\item urychlení procesu kalibrace,
	\item možnost provést dokalibrování za skutečného provozu,
	\item přehledné verzování projektu, čistý a přehledný kód, využití otevřených
	technologií.
\end{compactenum}

První bod vyplývá z~faktu, že proces kalibrace bude často spouštěn laickou obsluhou
kolejiště. Proto bude kladen důraz na očekávatelné chování grafického
rozhraní a jednoduchost práce se samotným měřicím hardwarem. Řešení bude také
navrženo tak, aby využívalo běžně dostupné technologie, se kterými jsou
uživatelé zvyklí pracovat.

Protože ve světě a v~Klubu odkazovaném v~kapitole \ref{sec:mod-kol} je
provozováno více modelových měřítek, autor se vynasnaží navrhnout řešení tak,
aby bylo nezávislé na měřítku.

Za okomentování dále stojí bod (5). Po zakoupení nového vozidla se
typicky provádí jeho kalibrace, vozidlo je následně začleněno do provozu, kde
jezdí i~několik let. Fyzikální realita světa je bohužel neúprosná --
ve vozidle dochází k~mechanickému opotřebení prvků, zaběhnutí pohyblivých
částí po určité době provozu a ke změně mechanických parametrů vozidla
v~závislosti na teplotě. Důsledkem těchto skutečností je, že kalibrace vozidla
po určité době ztrácí přesnost. Vychýlení typicky není zásadní, ale mnohdy
znatelné.

Proto je vhodné navrhnout kalibrační hardware tak, aby jej bylo možné použít
i~v~ostrém provozu na kolejišti. Vozidlo, které by mělo nedokonalou kalibraci,
by bylo dispečerem označeno a ovládací SW kolejiště by se postaral o~to, aby byla
kalibrace obnovena. Tato práce si sice neklade za cíl automatické
dokalibrovávání implementovat -- protože pro tento úkol bude nutná netriviální
součinnost s~řídícím SW kolejiště -- bylo by ale nanejvýš vhodné, aby na tento
proces bylo hardwarové řešení připraveno.


\chapter{Přehledová kapitola} \label{chap:prehled}
V~této kapitole budou stručně představena existující řešení problému měření
rychlosti modelového hnacího vozidla doplněná o~komentáře vhodnosti
jednotlivých řešení pro problém řešený touto prací. Dále budou představeny
dostupné technologie senzorů, které jsou pro problém měření rychlosti vhodné.

\section{Statické bodové detektory}

Jednou z~nejjednodušších technologií pro detekci rychlosti pohybujícího se
objektu jsou dva senzory detekující průchod bodem a stopky. V~naší aplikaci na
modelovém kolejišti si lze představit dva (např. optické) senzory umístěné
v~určité vzdálenosti od sebe, skutečnou rychlost pohybujícího se objektu lze
pak vypočíst jednoduchým vydělením vzdálenosti senzorů a času, po který se
hnací vozidlo pohybovalo mezi senzory.

Měření vychází z~předpokladu, že se hnací vozidlo pohybuje konstantní
rychlostí (podobně jako řada dalších rozebíraných metod měření níže), tento
předpoklad je však zejména díky kontrole nad ovládáním vozidla a díky
BEMF\footnote{Back Electro-motive force, technologie pro měření rychlosti
ovládaného motoru, viz \url{https://dccwiki.com/Back\_EMF}.}
možné bez problému splnit.

Konkrétní instancí této technologie je \textit{Model Railroad
Speedometer Accutrack II} \cite{accutrackII}.

Nespornými výhodami tohoto přístupu jsou jednoduchost a absence pohyblivých
prvků, která vede k~delší životnosti a dlouhodobé spolehlivosti měřiče.
Bohužel, zásadní nevýhodou tohoto řešení je nutnost měřit rychlost pouze ve
vybraném úseku tratě. V~požadavcích na řešení vyvíjené v~této práci je sice
specifikováno, že kalibrace bude probíhat na uzavřeném okruhu, avšak představme
si, že pro jedno změření rychlosti by bylo třeba buď projet celý okruh, nebo se
za senzorem zastavit a obrátit směr. Pomineme-li fakt, že motory často dávají
při stejné střídě napájecího signálu mírně rozdílné otáčky v~závislosti na
směru, ve kterém se pohybují, je i~tak čas nutný na celou kalibraci
nesrovnatelně vyšší než u~senzoru, který by byl schopen odečítat rychlost
průběžně, nezávisle na poloze vozidla.

Uvědomme si ale na závěr, že jedna z~klíčových nevýhod tohoto senzoru je
v~určitém slova smyslu i~jeho výhodou. Totiž fakt, že měření probíhá jen na pevně
definovaném úseku koleje s~sebou nese tu výhodu, že vozidlo měříme vždy ve
stejných podmínkách. Nemusíme tedy například řešit, jestli vozidlo zrovna jede
v~oblouku a je zpomalováno odstředivou silou, protože senzor prostě umístíme
na rovný úsek trati.

\section{Statické úsekové detektory}

Podobný přístup, jako u~statických bodových detektorů, nabízejí statické
úsekové detektory. Rozdíl oproti senzorům popsaným v~předchozí kapitole je
ve způsobu měření: senzory neměří projetí vozidla daným bodem, ale přítomnost
vozidla na úseku tratě určité délky.

Typickým zástupcem této technologie je proudový detektor obsazení, konkrétně
například \textit{BD20} \cite{bd20}.

Proudové detektory obsazení jsou hojně využívány na modelových kolejištích k~detekci
volnosti úseků a tudíž k~umožnění bezkolizního pohybu vozidla po trati.
Obdobné systémy se využívají na skutečné železnici, avšak tam jsou známy pod
názvem \textit{kolejové obvody}.

Konkrétní způsob měření přehledně popisují například tvůrci SW \textit{JMRI}
\cite{jmri:speedometer} (\textit{A Java Model Railroad Interface}).
Kalibrační kolej je rozdělena na několik částí (optimálně na tři části), kalibrační
SW pak postupně projíždí těmito částmi a počítá čas, za který vozidlo projede
prostřední částí. U~této části má od uživatele zadanou její délku, z~které pak
vypočte rychlost vozidla.

Výhody a nevýhody tohoto způsobu měření jsou víceméně stejné jako výhody a
nevýhody statických bodových detektorů. Malou výhodou úsekových detektorů je
fakt, že kolejiště často bývají vybavena úsekovými detektory pro účely
zapezpečovacího zařízení, takže není nutný nákup a instalace dalších senzorů.

\section{Podvalníkové detektory}

Zajímavým způsobem měření rychlosti prezentovaným v~\cite{bachrus}
je možnost měřit rychlost hnacího vozidla bez nutnosti pohybu po skutečné trati.

Tuto metodu si lze představit podobně jako měření rychlostí aut v~autoservisu.
Lokomotiva je položena na válce, které se pod ní mohou volně pohybovat.
Lokomotiva svým motorem roztáčí válce, jejichž rychlost je následně měřena.

Zajímavými výhodami této technologie jsou použitelnost prakticky pro jakékoliv
modelové měřítko, nezávislost na pozici vozidla na trati a v~neposlední řadě
zmizení nutnosti pořizovat kalibrační okruh. Celý proces kalibrace tak může
probíhat na velice malém prostoru.

Je rozhodně nutné uznat, že toto řešení je svým důvtipem minimálně oceněníhodné,
avšak pro potřeby této práce skrývá jednu zásadní nevýhodu. Totiž nemožnost
dokalibrovávat vozidlo za reálného provozu bez nutnosti uvolnění vozidla
z~reálného provozu na trati.

Stojí za to také poznamenat, že řešení, které by umožňovalo měřit rychlost
vozidla na kolejišti by s~sebou neslo tu výhodu, že jej lze použít k~měření
celkové ujeté vzdálenosti hnacího vozidla za delší dobu, což je údaj využitelný
například k~výpočtu míry opotřebení lokomotivy.

\section{Kalibrační SW}

Druhou, zcela zásadní částí této práce, je mimo tvorby senzoru pro měření
rychlosti hnacího vozidla také vytvoření programu, který zvládne celý proces
kalibrace automaticky -- bez zásahu uživatele. V~tomto hledisku autor bohužel
nemůže nabídnout výčet existujících řešení, protože taková řešení
prakticky neexistují. Je to z~toho důvodu, že většina dostupných SW pro řízení
modelového kolejiště staví na odlišném principu, než na kterém staví řídicí SW
hJOP \cite{hjop:web}, pro který je automatická kalibrace řešena.

Většina obslužných programů totiž počítá s~tím, že hnací vozidla na kolejišti
jsou různá a nesnaží se je jednotně kalibrovat. Místo toho si tyto SW před
prvním použitím lokomotivy nechají zadat její parametry nebo si je změří jednou
z~technologií popsaných výše a u~lokomotivy si tzv. \textit{kalibrační tabulku}
dlouhodobě udržují. Tyto programy tak kompenzují rozdílné vlastnosti vozidel přímo
povelováním různými rychlostmi.

Koncepce SW hJOP je však jiná. Bylo rozhodnutím autora, že je vhodnější mít
zkalibrovaná vozidla na úrovni dekodérů, než si v~SW ukládat pro každé hnací
vozidlo speciální \textit{kalibrační tabulku}.

\section{Přehled dostupných senzorů}

\textit{TODO}


\printbibliography[heading=bibintoc]

\appendix
\chapter{Appendix}
Appendix.

\end{document}
