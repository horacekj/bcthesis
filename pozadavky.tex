Pro návrh vhodného řešení procesu automatické kalibrace je nutné formulovat
požadavky kladené na toto řešení. Základní požadavky seřazené od
nejdůležitějšího jsou:

\begin{compactenum}
	\item snadné použití pro koncového uživatele, snadná instalace,
	\item možnost použít řešení pro libovolné modelové měřítko,
	\item maximalizace automatizace a minimalizace nutné spolupráce s~uživatelem,
	\item urychlení procesu kalibrace,
	\item možnost provést dokalibrování za skutečného provozu,
	\item přehledné verzování projektu, čistý a přehledný kód, využití otevřených
	technologií.
\end{compactenum}

První bod vyplývá z~faktu, že proces kalibrace bude často spouštěn laickou obsluhou
kolejiště. Proto bude kladen důraz na očekávatelné chování grafického
rozhraní a jednoduchost práce se samotným měřicím hardwarem. Řešení bude také
navrženo tak, aby využívalo běžně dostupné technologie, se kterými jsou
uživatelé zvyklí pracovat.

Protože ve světě a v~Klubu odkazovaném v~kapitole \ref{sec:mod-kol} je
provozováno více modelových měřítek, autor se vynasnaží navrhnout řešení tak,
aby bylo nezávislé na měřítku.

Za okomentování dále stojí bod (5). Po zakoupení nového vozidla se
typicky provádí jeho kalibrace, vozidlo je následně začleněno do provozu, kde
jezdí i~několik let. Fyzikální realita světa je bohužel neúprosná --
ve vozidle dochází k~mechanickému opotřebení prvků, zaběhnutí pohyblivých
částí po určité době provozu a ke změně mechanických parametrů vozidla
v~závislosti na teplotě. Důsledkem těchto skutečností je, že kalibrace vozidla
po určité době ztrácí přesnost. Vychýlení typicky není zásadní, ale mnohdy
znatelné.

Proto je vhodné navrhnout kalibrační hardware tak, aby jej bylo možné použít
i~v~ostrém provozu na kolejišti. Vozidlo, které by mělo nedokonalou kalibraci,
by bylo dispečerem označeno a ovládací SW kolejiště by se postaral o~to, aby byla
kalibrace obnovena. Tato práce si sice neklade za cíl automatické
dokalibrovávání implementovat -- protože pro tento úkol bude nutná netriviální
součinnost s~řídícím SW kolejiště -- bylo by ale nanejvýš vhodné, aby na tento
proces bylo hardwarové řešení připraveno.
