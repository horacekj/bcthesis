Pro návrh vhodného řešení procesu automatické kalibrace je nutné si ujasnit
požadavky kladené na toto řešení. Základní požadavky seřazené od
nejdůležitějšího jsou:

\begin{enumerate}
	\item snadné použití pro koncového uživatele, snadná instalace,
	\item možnost použít řešení pro libovolné modelové měřítko,
	\item maximalizace automatizace a minimalizace nutné spolupráce s~uživatelem,
	\item urychlení procesu kalibrace,
	\item možnost provést dokalibrování za skutečného provozu,
	\item slušné verzování projektu, přehledný a čistý kód, využití otevřených
	technologií.
\end{enumerate}

První bod vyplývá z~faktu, že proces kalibrace bude spouštěn běžnou obsluhou
kolejiště, která často nemá kapacity na to, aby se seznamovala s~novými
technologiemi. Proto bude kladen důraz na očekávatelné chování grafického
rozhraní a jednoduchost práce se samotným měřícím hardwarem. Řešení bude také
navrženo tak, aby využívalo běžně dostupné technologie, se kterými jsou
uživatelé zvyklí pracovat.

Protože v~Klubu odkazovaném výše je provozováno více modelových měřítek, autor
se vynasnaží navrhnout řešení tak, aby bylo nezávislé na měřítku.

Za okomentování pak určitě stojí bod (5). Po zakoupení nového vozidla se
typicky provádí jeho kalibrace, vozidlo je následně začleněno do provozu, kde
i~několik let jezdí. Fyzikální realita světa je bohužel neúprosná v~tom směru,
že ve vozidle dochází k~mechanickému opotřebení prvků, zaběhnutí pohyblivých
částí po určité době provozu a ke změně mechanických parametrů vozidla
v~závislosti na teplotě. Tyto skutečnosti vedou k~tomu, že kalibrace vozidla
je po určité době provozu narušena. Vychýlení typicky není zásadní, ale často
znatelné.

Proto by bylo vhodné navrhnout kalibrační hardware tak, aby šel použít
i~v~ostrém provozu na kolejišti. Vozidlo, které by mělo narušenou kalibraci,
by bylo dispečerem označeno a ovládací SW by se postaral o~to, aby byla
kalibrace obnovena. Tato práce si rozhodně neklade za cíl automatické
dokalibrovávání implementovat, protože pro tento úkol bude nutná netriviální
součinnost s~řídícím SW kolejiště. Bylo by ale nanejvýš vhodné, aby na tento
proces bylo hardwarové řešení připraveno.
