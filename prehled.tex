V~této kapitole budou stručně představena existující řešení problému měření
rychlosti modelového hnacího vozidla doplněná o~komentáře vhodnosti
jednotlivých řešení pro problém řešený touto prací. Dále budou představeny
dostupné technologie senzorů, které jsou pro problém měření rychlosti vhodné.

Ze senzorů pro měření rychlosti je obvykle snadno možné dopočítat dráhu, proto
speciální senzory pro měření dráhy nebudou uvažovány.

\section{Statické bodové detektory}
\label{sec:det-static}

Jednou z~nejjednodušších technologií pro detekci rychlosti pohybujícího se
objektu jsou dva senzory detekující průchod bodem a stopky. V~naší aplikaci na
modelovém kolejišti si lze představit dva (např. optické) senzory umístěné
v~určité vzdálenosti od sebe, skutečnou rychlost pohybujícího se objektu lze
pak vypočíst jednoduchým vydělením vzdálenosti senzorů a času, po který se
hnací vozidlo pohybovalo mezi senzory.

Měření vychází z~předpokladu, že se hnací vozidlo pohybuje konstantní
rychlostí (podobně jako řada dalších rozebíraných metod měření níže), tento
předpoklad je však zejména díky kontrole nad ovládáním vozidla a díky
BEMF\footnote{Back Electro-motive force, technologie pro měření rychlosti
ovládaného motoru, viz \url{https://dccwiki.com/Back\_EMF}.}
možné bez problému splnit.

Konkrétní instancí této technologie je \textit{Model Railroad
Speedometer Accutrack II} \cite{accutrackII}.

Nespornými výhodami tohoto přístupu jsou jednoduchost a absence pohyblivých
prvků, která vede k~delší životnosti a dlouhodobé spolehlivosti měřiče.
Bohužel, zásadní nevýhodou tohoto řešení je nutnost měřit rychlost pouze ve
vybraném úseku tratě. V~požadavcích na řešení vyvíjené v~této práci je sice
specifikováno, že kalibrace bude probíhat na uzavřeném okruhu, avšak představme
si, že pro jedno změření rychlosti by bylo třeba buď projet celý okruh, nebo se
za senzorem zastavit a obrátit směr. Pomineme-li fakt, že motory často dávají
při stejné střídě napájecího signálu mírně rozdílné otáčky v~závislosti na
směru, ve kterém se pohybují, je i~tak čas nutný na celou kalibraci
nesrovnatelně vyšší než u~senzoru, který by byl schopen odečítat rychlost
průběžně, nezávisle na poloze vozidla.

Uvědomme si ale na závěr, že jedna z~klíčových nevýhod tohoto senzoru je
v~určitém slova smyslu i~jeho výhodou. Totiž fakt, že měření probíhá jen na pevně
definovaném úseku koleje s~sebou nese tu výhodu, že vozidlo měříme vždy ve
stejných podmínkách. Nemusíme tedy například řešit, jestli vozidlo zrovna jede
v~oblouku a je zpomalováno odstředivou silou, protože senzor prostě umístíme
na rovný úsek trati.

\section{Statické úsekové detektory}
\label{sec:det-usek}

Podobný přístup, jako u~statických bodových detektorů, nabízejí statické
úsekové detektory. Rozdíl oproti metodě popsané v~předchozí kapitole je
ve způsobu měření: senzor neměří projetí vozidla daným bodem, ale přítomnost
vozidla na úseku tratě určité délky.

Typickým zástupcem této technologie je proudový detektor obsazení, konkrétně
například \textit{BD20} \cite{bd20}.

Proudové detektory obsazení jsou hojně využívány na modelových kolejištích k~detekci
volnosti úseků a tudíž k~umožnění bezkolizního pohybu vozidla po trati.
Obdobné systémy se využívají na skutečné železnici, avšak tam jsou známy pod
názvem \textit{kolejové obvody}.

Konkrétní způsob měření přehledně popisují například tvůrci SW \textit{JMRI}
\cite{jmri:speedometer} (\textit{A Java Model Railroad Interface}).
Kalibrační kolej je rozdělena na několik částí (optimálně na tři části),
kalibrační SW pak lokomotivou postupně projíždí těmito částmi a počítá čas, za
který vozidlo projede prostřední částí. U~této části má od uživatele zadanou
její délku, z~které pak vypočte rychlost vozidla.

Výhody a nevýhody tohoto způsobu měření jsou víceméně stejné jako výhody a
nevýhody statických bodových detektorů. Malou výhodou úsekových detektorů je
fakt, že kolejiště často bývají vybavena úsekovými detektory pro účely
zapezpečovacího zařízení, takže není nutný nákup a instalace dalších senzorů.

\section{Podvalníkové detektory}
\label{sec:det-podval}

Zajímavým způsobem měření rychlosti prezentovaným v~\cite{bachrus}
je možnost měřit rychlost hnacího vozidla bez nutnosti pohybu po skutečné trati.

Tuto metodu si lze představit podobně jako měření rychlostí aut v~autoservisu.
Lokomotiva je položena na válce, které se pod ní mohou volně pohybovat.
Lokomotiva svým motorem roztáčí válce, jejichž rychlost je následně měřena.

Zajímavými výhodami této technologie jsou použitelnost prakticky pro jakékoliv
modelové měřítko, nezávislost na pozici vozidla na trati a v~neposlední řadě
zmizení nutnosti pořizovat kalibrační okruh. Celý proces kalibrace tak může
probíhat na velice malém prostoru.

Je rozhodně nutné uznat, že toto řešení je svým důvtipem minimálně oceněníhodné,
avšak pro potřeby této práce skrývá jednu zásadní nevýhodu. Totiž nemožnost
dokalibrovávat vozidlo za reálného provozu bez nutnosti uvolnění vozidla
z~reálného provozu na trati.

Stojí za to také poznamenat, že řešení, které by umožňovalo měřit rychlost
vozidla na kolejišti za provozu by s~sebou neslo tu výhodu, že jej lze použít
k~měření celkové ujeté vzdálenosti hnacího vozidla za delší dobu, což je údaj
využitelný například k~výpočtu míry opotřebení lokomotivy.

\section{Kalibrační SW}
\label{sec:kalib-sw}

Druhou, zcela zásadní částí této práce, je mimo tvorby senzoru pro měření
rychlosti hnacího vozidla také vytvoření programu, který zvládne celý proces
kalibrace automaticky -- bez zásahu uživatele. V~tomto hledisku autor bohužel
nemůže nabídnout výčet existujících řešení, protože taková řešení
prakticky neexistují. Je to z~toho důvodu, že většina dostupných SW pro řízení
modelového kolejiště staví na odlišném principu, než na kterém staví řídicí SW
hJOP \cite{hjop:web}, pro který je automatická kalibrace řešena.

Většina obslužných programů totiž počítá s~tím, že hnací vozidla na kolejišti
jsou různá a nesnaží se je jednotně kalibrovat. Místo toho si tyto SW před
prvním použitím lokomotivy nechají zadat její parametry nebo si je změří jednou
z~technologií popsaných výše a u~lokomotivy si tzv. \textit{kalibrační tabulku}
dlouhodobě udržují. Tyto programy tak kompenzují rozdílné vlastnosti vozidel přímo
povelováním různými rychlostními stupni.

Koncepce SW hJOP je však jiná. Bylo rozhodnutím autora, že je vhodnější mít
zkalibrovaná vozidla na úrovni dekodérů, než si v~SW ukládat pro každé hnací
vozidlo speciální \textit{kalibrační tabulku}.

\section{Závěr}
\label{sec:prehled-zaver}

Z~popisu vlastností výše uvedených komerčně dostupných technologií měření
rychlosti modelového hnacího vozidla vyplývá, že žádné z~těchto řešení
nevyhovuje autorovým požadavkům plně. Proto autor přistoupí k~návrhu vlastního
měřicího systému.
