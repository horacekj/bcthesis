Tato práce ukázala, že k~automatizaci procesu kalibrace modeového železničního
vozidla je nutné zvládnout širokou škálu dílčích kroků, od efektivního návrhu
elektroniky, přes programování mikrokontrolérů, studování bezdrátových modulů
a~komunikačních protokolů až po schopnost kvalitně vyvíjet desktopové aplikace.
Autorovi se úspěšně povedlo navrhnout vlastní senzor pro měření rychlosti
založený na optozávoře a~perforovaném kolu. Podařilo se mu zabezpečit přenos
naměřených dat do počítače pomocí univerzálního protokolu Bluetooth a
demonstrovat fungování přenosu. Jeho práci pak dovršila kompletně vlastní
aplikace, která umožňuje proces kalibrace realizovat pohodlně a~spolehlivě.
Kromě toho byla vytvořena knihovna, která je univerzálně použitelná pro
komunikaci s~libovolnou XpressNET centrálou.

Při konstrukci měřicího vozu vyvstalo několik problémů a~omezení souvisejících
s~omezeným prostorem a~limitovanými možnostmi napájení. Ukázalo se, že je možné
minimalizovat spotřebu zařízení tak, aby jej bylo možné napájet z~baterie.
Napájení z~baterie se nakonec ukázalo jako nanejvýš výhodné a~jednoduché.

Bylo zjištěno, že rychlost vozidla je poměrně zásadně ovlivňována nejrůznějšími
parazitickými jevy, od pružení ve spřáhle, přes odpor v~oblouku až po mechaniku
převodů v~lokomotivě. Tyto jevy se autorovi povedlo úspěšně odstínit za použití
průměrování.

Tato práce jednoznačně pomohla k~zefektivnění a~zpřesnění procesu kalibrace
lokomotivy, který místo 25 minut za využití jednoho člověka trvá 4 minuty
bez nutnosti přítomnosti obsluhy.

\section{Možná rozšíření}

Při návrhu senzoru byl brán v~potaz požadavek na možnost dokalibrovávat
lokomotivy na ostrém provozu na kolejišti. Nabízí se tedy možnost prozkoumat,
jak moc se rychlosti lokomotiv vlivem provozu a opotřebení mění a za využití
kalibračního vozu navrhnout mechanismus, kterým budou hnací vozidla
dokalibrována za provozu, nejlépe ve spolupráci s~řídicím SW kolejiště.

Dalším směrem výzkumu je měření příčin oscilace rychlosti a návrh
aktivního potlačení těchto oscilací. Systém efektivního potlačování výkyvů
rychlostí by sice nejspíš byl výpočetně náročný (Fourierova transformace,
konvoluční metody), ale vzhledem k~výkonům současných počítačů uskutečnitelný.
A~to možná i~v~reálném čase. Uživateli by tak byla nabídnuta vysoce přesná šumu
zbavená data v~reálném čase. To by mohlo přispět ke zrychlení všech procesů
založených na měření rychlosti měřicím vozem.
