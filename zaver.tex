Tato práce demonstrovala, že k automatizaci procesoru kalibrace železničního
modelového vozidla je nutné zvládnout širokou škálu dílčích kroků, od
efektivního návrhu vlastní elektroniky, přes znalost fungování mikrokontrolérů,
bezdrátových modulů a komunikačních protokolů až po schopnost kvalitně vyvíjet
desktopové aplikace. Autorovi se úspěšně povedlo navrhnout vlastní senzor pro
měření rychlosti založený na optozávoře a perforovaném kolu a podařilo se mu
zabezpečit přenos naměřených dat do počítače pomocí univerzálního protokolu
Bluetooth. Jeho práci pak dovršila kompletně vlastní aplikace, která umožňuje
proces kalibrace realizovat pohodlně a spolehlivě. Kromě toho byla vytvořena
knihovna, která je univerzálně použitelná pro komunikaci s libovolnou XpressNET
centrálou.

Při konstrukci měřicího vozu vyvstalo několik problémů a omezení souvisejících
s omezeným prostorem a možnostmi napájení. Ukázalo se, že je možné minimalizovat
spotřebu zařízení tak, aby jej bylo možné napájet z baterie. Napájení z baterie
se nakonec ukázalo jako nanejvýš výhodné a jednoduché.

Bylo zjištěno, že rychlost vozidla je poměrně zásadně ovlivňována nejrůznějšími
parazitickými jevy, od pružení ve spřáhle, přes odpor v oblouku až po mechaniku
převodů v lokomotivě. Tyto jevy se autorovi povedlo úspěšně odstínit za použití
průměrování.

Tato práce jednoznačně pomohla k zefektivnění a zpřesnění procesu kalibrace
lokomotivy, který místo 25 minut za využití jednoho člověka trvá 4 minuty
bez nutnosti přítomnosti obsluhy.

\section{Náměty na rozšíření}


