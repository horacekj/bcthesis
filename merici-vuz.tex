Tato kapitola popisuje celkovou koncepci vlastního senzoru pro měření
rychlosti hnacího vozidla -- tzv. měřicího vozu. Tato kapitola popisuje
a zdůvodňuje rozhodnutí, která autor při konstrukci senzoru zvolil.

V~kontextu této kapitoly si autor dovoluje stručně připomenout Požadavky
na řešení (TODO: link chapter 2).

\section{Umístění senzoru}

Uvědomme si nejprve, kam je a není možno senzor umístit. Určitě platí, že
senzor není možné umístit do lokomotivy. Lokomotiva je totiž komerční produkt,
který umožňuje úpravy elektroniky uvnitř nejvýše na úrovní výměny digitálního
dekodéru DCC. Lokomotiva obvykle bývá vybavena deskou plošných spojů, která
připojuje jednotlivé součásti lokomotivy (motor, koleje, dekodér, osvětlení,
...). Tato deska je specifická pro konkrétní model a dodává ji výrobce hnacího
vozidla. Není tedy prakticky možné do elektroniky lokomotivy nějak zasáhnout.

Další možností je doplnit senzor do lokomotivy bez úpravy její elektroniky.
To s sebou ale nese pořád tu nevýhodu, že je třeba mechanicky zasahovat do
(poměrně drahé) lokomotivy.

Další možností je umístit senzor mimo hnací vozidlo. Tento senzor může být
umístěn na kolejiště staticky (podobně jako v \link{senzor}) nebo může
být součástí vagónu připojeného za lokomotivou. Statický senzor se jeví jako
nevyhovující zejména proto, že neumožňuje provádění dokalibrace za skutečného
provozu.

Nejvhodnějším řešením se tedy jeví vyrobit vůz, který se připojí ke kalibrované
lokomotivě jako běžný vagón. Součástí tohoto vozu bude senzor pro měření
rychlosti. Vůz pak bude možné připojit bez narušení provozu i~za vlak na
produkčním kolejišti.

S uvážením výše uvedených argumentů se autor rozhodl vydat cestou
\textit{měřicího vozu}.

\section{Dostupné technologie senzorů}



