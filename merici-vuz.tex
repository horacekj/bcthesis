Tato kapitola popisuje celkovou koncepci vlastního senzoru pro měření
rychlosti hnacího vozidla -- tzv. měřicího vozu. Tato kapitola popisuje
a zdůvodňuje rozhodnutí, která autor při konstrukci senzoru zvolil.

V~kontextu této kapitoly si autor dovoluje stručně připomenout Požadavky
na řešení (TODO: link chapter 2).

\section{Umístění senzoru}

Uvědomme si nejprve, kam je a není možno senzor umístit. Určitě platí, že
senzor není možné umístit do lokomotivy. Lokomotiva je totiž komerční produkt,
který umožňuje úpravy elektroniky uvnitř nejvýše na úrovní výměny digitálního
dekodéru DCC. Lokomotiva obvykle bývá vybavena deskou plošných spojů, která
připojuje jednotlivé součásti lokomotivy (motor, koleje, dekodér, osvětlení,
...). Tato deska je specifická pro konkrétní model a dodává ji výrobce hnacího
vozidla. Není tedy prakticky možné do elektroniky lokomotivy nějak zasáhnout.

Další možností je doplnit senzor do lokomotivy bez úpravy její elektroniky.
To s sebou ale nese pořád tu nevýhodu, že je třeba mechanicky zasahovat do
(poměrně drahé) lokomotivy.

Další možností je umístit senzor mimo hnací vozidlo. Tento senzor může být
umístěn na kolejiště staticky (podobně jako v \ref{senzor}) nebo může
být součástí vagónu připojeného za lokomotivou. Statický senzor se jeví jako
nevyhovující zejména proto, že neumožňuje provádění dokalibrace za skutečného
provozu.

Nejvhodnějším řešením se tedy jeví vyrobit vůz, který se připojí ke kalibrované
lokomotivě jako běžný vagón. Součástí tohoto vozu bude senzor pro měření
rychlosti. Vůz pak bude možné připojit bez narušení provozu i~za vlak na
produkčním kolejišti.

S uvážením výše uvedených argumentů se autor rozhodl vydat cestou
\textit{měřicího vozu}.

\section{Senzor}

Hlavní součásti měřicího vozu je nesporně rychlostní senzor. Zvolená
technologie senzoru má vliv na formát a přesnost měření, dále má vliv na
mechanicko-elektrické vlastnosti měřicího vozu, jeho cenu, snadnost výroby,
opakovatelnost výroby s výhledem na několik let a v neposlední řadě poruchovost
a servisovatelnost měřicího vozu.

Autor uvážil několik dostupných technologií senzorů.

\subsection{Magnetický senzor}

První možnou technologií senzoru jsou tzv. \textit{magnetické poziční senzory}.
Tyto senzoru fungují na bázi halových sond. Umí měřit úhel natočení objektu
vzhledem k senzoru. Jejich primárním účelem je měřit natočení nejrůznějších
ramen, či prvků ovládaných např. servomotory, opakovaným měřením natočení lze
ale dosáhnout velice přesného měření rychlosti. Princip těchto senzorů
dobře ilustruje obrázek \ref{fig-magnetic-sensor}.

TODO obrázek.

Výhodou tohoto měření je velice vysoká přesnost (řádově 10 bitů na jednu otáčku
-- TODO je to pravda?). Bohužel, při pohledu na obrázek
\ref{fig-magnetic-sensor} je velice těžko představitelné, jak umístit senzor na
nápravu vagónu. Senzor by musel být umístěn z boku vozu, což by výrazně
snižovalo jeho stabilitu.

Existují ale i senzory, které se montují přímo na osu rotujícího předmětu
(TODO: link). Takové by pro náš účel byly nejvhodnější. Smutným faktem ale je,
že tyto senzory vyžadují magnet o~tak velké velikosti, že jsou pro modelovou
železnici nepoužitelné.

\subsection{Magnetický senzor \uv{cykloměřič}}

Technologie tzv. \uv{cykloměřiče} byla historicky první metodou jak měřit
rychlost modelové lokomotivy. Tento senzor jsem dokonce viděl v praxi, je
založený na stejné principu, jako počítač rychlosti na cyklistickém kole.
Na osu nápravu je připevněn magnet, ve voze je pak Hallova sonda. Magnet udělá
na jednu otáčku jeden zákmit na Hallově sondě, počítač pak počítá počty zákmitů.

Tato možnost je poměrně jednoduchá na instalaci, množství mechanických prvků
je poměrně malé, na druhou stranu ale poskytuje poměrně malou přesnost --
zejména při nižších rychlostech. Tato technologie senzoru se tedy pro potřeby
této práce také nejeví jako nejvhodnější.

\subsection{Optický senzor}

Třetí uvažovanou technologií senzoru je optický senzor, který je každému
důvěrně známý z optických myší. Tento senzor pracuje na principu snímání obrazu
podložky CCD čipem a porovnávání dvou obrazů pořízených v krátkém časovém
intervalu. Ze dvou obrazů je pak určen rozdíl polohy myši.
\footnote{Zajímavostí je, že celá logika porovnávání dvou obrazů je
v~takovýchto senzorech implementována hardwarově.}

Toto řešení vypadá velice slibně, jeho obrovskou výhodou je absence jakýchkoliv
mechanických částí. Stačí namířit senzor na pražce a prostě měřit jak se vůz
pohybuje. Pražce jsou navíc velice dobře kontrastní vůči štěrku, který je mezi
nimi, porovnávání obrazů by tak mohlo být velice přesné (srovnejte s~pohybem
optické myši po takřka uniformním povrchu stolu nebo podložky pod myš).

Pro kalibrační vůz byl vybrán konkrétní senzor ADNS-3050. Kritériem výběru
bylo to, že (1) senzor má výstup sériovou linkou a že (2) senzor zvládá měřit
vysoké rychlosti (vagón se typicky pohybuje o~něco rychleji než myš).

Autor nakoupil senzory, osadil je na desku plošných spojů a začal testovat
přesnost měření pomocí jednoduchého zapojení s vývojovým kitem Arduino. Velikou
výzvou bylo sehnat čočku, která autorovi umožní zaměřit pražce a zároveň
ponechat senzor na ploše vagónu. Díky spolupráci s Ústavem přístrojové techniky
AV ČR se však povedlo dostatečně malou čočku úspěšně sehnat.

První měření ukázala, že pro dobrou funkci senzoru je třeba poměrně kontrastní
snímaná plocha a poměrně intenzivní červené světlo (na to je CCD čip v~senzoru
nejcitlivější \footnote{Proto si optická myš přisvětluje červeným světlem.})
Navíc se ukázalo, že hloubka ostrosti senzoru je nejvýše $1\ mm$, což reálně
znemožňovalo měřit rychlosti na kolejích, které nejsou zasypány štěrkem, ale
pouze připevněny k~podložce. Složitý optický senzor tak degradoval na zařízení,
jehož funkci by v podstatě šlo zařídit pomocí jedné LED diody a jednoho
fototranzistoru \footnote{Stačí jednoduše měřit intenzitu odraženého světla a
zkoumat, jestli je senzor zrovna na pražci nebo mezi pražci.}.

Po několika měsících pokusů o zprovoznění měření optickým senzorem autor naznal,
že s tímto typem senzoru nebude schopen garantovat přesně a spolehlivé měření.
Padlo tedy rozhodnutí tuto technologii opustit. Autor by rád zdůraznil, že
jednou z hlavních motivací pro opuštění této technologie byla přílišná
komplexnost toho, co se děje v senzoru, a neschopnost ovlivňování a debugování
těchto procesů.

TODO: images, link ADNS sensor

\subsection{Optozávora}

Třetí uvažovaná technologie senzorů je inspirována návrhem od kolegy modeláře
Petra Trávníka. Základem této technologie je paprskové kolo \ref{wheel} upevněné
na nápravě měřicího vozu. Paprskové kolo se otáčí spolu s nápravou, k vozu je
pak připevněna optozávora, která snímá průchod jednotlivých paprsků. Klíčovým
faktorem pro použitelnost této technologie jsou zejména technické možnosti
výroby dostatečného množství paprsků při zajištění rovnoměrnosti rozestupu
paprsků a malé velikosti kola. Průměr kola by totiž neměl přesáhnout $6\ mm$
(měřítko TT, 1:120).

Ze všech uvažovaných technologií autor nakonec zvolil právě technologii
paprskového kola a optozávory.

Nejvhodnější technologií pro výrobu takového kola je zřejmě leptání, protože
ale autor nemá k této technologii přístup, rozhodl je místo jednotlivých
paprsků kolo vyrobit pouze dírkované. Jako vhodný kompromis bylo stanoveno
8 dírek na kolo, přičemž chyba v~umístění jedné dírky byla výrobou stanovena na
nejvýše $0.05\ mm$ (TODO: ověřit). To jsou poměrně dobré výsledky.

Výkres kola je zobrazen na obrázku \ref{wheel}.

TODO: figure

Nemalým problémem pak bylo sehnat dostatečně malou optozávoru. Běžné optozávory
dostupné v~českých obchodech dosahují rozměrů cca $15\times6\times10\ mm$
(TODO: cite). To je pro účely vozu v~měřítku 1:120 zcela nepřijatelné. Po
průzkumu možných alternativ byl nakonec vybrán senzor GP1S23 (TODO link), který
byl dříve hojně montován d disketových mechanik. Dnes je senzor běžně dostupný
například na modulu k~Arduinu pod označením KY-010 (TODO link).

\section{Přenos dat do PC}

Dle požadavků na měřicí technologii\ref{pozadavky} je nutné zajistit bezdrátový
přenos dat ze senzoru. Primární technologií na straně příjemce je pro tento
projekt počítač, pokud se však podaří vybrat takovou technologii, aby data
mohla být přenášena například i na mobilní telefon či tablet, bude pro pro
měřicí vůz jen výhodou. Autor se proto rozhodl vyhnout se proprietárním řešením
pro bezdrátový přenos dat a cílit na technologie, které jsou běžně dostupné,
ideálně přímo zabudované v dnešních počítačích. Dalším argumentem pro toto
rozhodnutí je to, že nutnost pořizovat (natož stavět!) přijímač by zvyšoval
jak finanční tak výrobní náročnost celého projektu.

S tímto vědomím přicházejí v úvahu 2 technologie.

\begin{enumerate}
\item Wi-Fi
\item Bluetooth
\end{enumerate}

\section{Napájení}

Jedním z klíčových faktorů, který určoval výslednou podobu měřicího vozu, bylo
napájení. Pro napájení elektroniky jsou 3 možnosti.

\begin{enumerate}
\item Napájení z kolejí.
\item Napájení z baterie.
\item Kombinace předchozích dvou (místo baterie lze případně použít kondenzátory).
\end{enumerate}

Výhody a nevýhody jsou poměrně jasné: baterie má omezenou výdrž, napájení
z kolejí je jistota. Na tomto místě je ale třeba zmínit, že napětí v kolejích
sice je (téměř) trvale přítomné, ale zajistit jeho spolehlivé sbírání je
skutečně výzva. Sbírání je typicky realizováno plíšky s kontakty na kola.
Na základě odhadnuté spotřeby měřicího vozu (viz níže) byla vytvořena modelová
zátěž a indikace, jak dobře je sbírání schopno tuto zátěž \uv{uživit}.

Naměřená data bohužel byla naprosto tristní. Sbírání bylo realizováno skrze
hrotová ložiska, protože při použití sbíracích kontaktů přiložených přímo na
kolečko vzniká riziko, že tření mezi kontaktem a kolem bude tak vysoké, že kolo
bude na kolejnici prokluzovat. A to je v kontextu faktu, že na tomto kole
probíhá měření rychlosti, naprosto nemyslitelné. Zátěž byla živena cca $50 \%$
času a to i veškerou snahou o přídavné kondenzátory a případnou přídavnou zátěž
pro zlepšení kontaktu.

Na základě těchto měření i autorových předchozích zkušeností se sbíráním
do přípojných vozů se autor rozhodl, že tudy cesta nevede. Chvíli byla zvažována
řešení zahrnující baterie i sbírání, ale tato řešení se ukázala jako příliš
komplikovaná.

Poslední alternativou tedy zůstala baterie. Velkou výzvou při použití baterie
je vměstnat baterii do malého prostoru vozu a zároveň udržet co nejdelší výdrž.
Minimální doba kalibrace na jedno nabití byla autorem stanovena na 3 hodiny.
Výhodou baterie je navíc to, že není nutné vůz nijak upravovat (instalovat
sběrací kontakty) a vůz je za plné funkčnosti zařízení bez problému sejmout
z~kolejí a například tak vyzkoušet funkčnost senzoru.

Technologií baterií byla zvolena LiPol.

Skutečnou výzvou se tak stalo vyrobit elektroniku tak, aby její spotřeba
byla vskutku malá (viz další kapitolu).

\section{Platforma}

Na základě požadavků vyplynuvších z předchozích kapitol (senzor, přenos dat)
je nutné vybrat hardware, který bude umístěn do měřicího vozu.

\subsection{ESP-32}

Prvním autorovým pokusem byl procesor ESP-32\ref{esp-32}, jehož největší
výhodou je vestavěný Bluetooth a WiFi. Jedná se o poměrně výkonný procesor,
který lze taktovat až na 240 MHz a to dokonce na dvou jádrech!
\ref{esp-32-datasheet}

Autor pořídil ESP-32 vývojový kit a začal zkoušet možnosti komunikace. Jako
primární komunikační kanál byl vybrán Bluetooth, jelikož vyžaduje oproti
WiFi menší režii spojenou s ustanovením spojení. Tento problém lze řešit tím,
že se měřicí vůz nepřipojuje k již existující síti, ale sám síť vytváří,
negativem tohoto přístupu je však ale to, že není typicky možné na počítači,
který přijímá signál z měřícího vozu, být zároveň připojený do sítě internet
pomocí WiFi. A to je nanejvýš nepraktické.

Bluetooth na ESP-32 fungoval zdárně. Na ESP se podařilo aktivovat Bluetooth SPP
profil \ref{spp}, takže se zařízení poměrně snadno spárovalo s počítačem a
tvářilo se jako sériová linka. Profil SPP byl vybrán z toho důvodu, že má
poměrně dobrou implementaci na celé škále operačních systémů a vytváří dobrou
abstrakci pro použití z jakéhokoliv programu. Téměř na každém operačním systému
totiž umíte \uv{otevřít sériovou linku}. Kdyby se pak podařilo zvolit pro vývoj
aplikace takový framework, který zvládá abstrahovat sériový port nezávisle na
cílové platformě, měl by autor úplně vyhráno. Vyrobil by dokonalou
multiplatformní aplikaci.

Po nasazení profilu SPP se však ukázalo, že spotřeba ESP-32 je cca $80\ mA$,
což jej dělá -- v kontextu napájení z baterie -- nepoužitelným. Bylo zvažováno
využít profilu specifikace \textit{BLE (Bluetooth Low Energy)}, BLE však nemá
profil pro sériovou linku, takže by autor musel vytvořit profil vlastní. To
by znamenalo nutnost dodat specifikaci profilu do klientského zařízení (počítač)
a místo abstrakce sériovou linkou přistupovat přímo k zařízení Bluetooth.
Toto řešení bylo zamítnuto pro příliš velkou pracnost a ztrátu elegance.

Autor ještě chvíli zvažoval, jak tento problém řešit pomocí sbírání či zvětšení
baterie, všechna navrhovaná řešení však byla příliš komplikovaná nebo
nefungovala dobře. Po několika týdnech testování tak bylo ESP-32 zavrhnuto.

\subsection{ATmega + HC-05}

Autorovou druhou volbou byla jemu dobře známá rodina procesorů s architekturou
AVR názvu ATmega\ref{atmega}. Procesory rodiny ATmega jsou univerzální
nízkopříkonové procesory, které jsou základem například vývojové desky Arduino
\ref{arduino}. Volba na tuto rodinu procesorů padla také z toho důvodu, že
s jejím programováním měl autor předchozí zkušenosti.

Procesory ATmega v sobě ale bohužel nezahrnují modul pro bezdrátovou komunikaci.
Autor se tedy rozhodl využít externí Bluetooth modul propojený s procesorem.
Při výběru modulu padla jasná volba na běžně dostupný BT modul názvu HC-05.
Výhodou tohoto modulu je jeho vysoká dostupnost a nízká cena, nevýhodou je to,
že modul je -- lidově řečeno -- \uv{čínský}. To znamená, že k němu není až
tak dobrá dokumentace a v Česku je to s jeho dostupností horší. Každopádně
co se týče světové dostupnosti, je na tom modul velice dobře. Autor zvažoval
ještě další moduly (HC-06, HC-07), tyto moduly by však pro účel této práce
nepřinesly žádnou zásadní výhodu.

Autor experimentálně změřit, že spotřeba nespárovaného HC-05 modulu je
$TODO\ mA$, spotřeba spárovaného modulu je $TODO\ mA$. To jsou oproti ESP-32
výrazně příjemnější čísla. Autorovi se navíc podařilo zajistit spolehlivou
funkci senzoru již při proudu $10\ mA$, což je polovina proudu, na který
je senzor dimenzován\ref{gp1s23-datasheet}. S přihlédnutím ke spotřebě procesoru
a drobných součástí jako LE diod by se měla pohybovat celková spotřeba celého
zařízení kolem cca $30\ mA$. Tato spotřeba byla uznána jako přijatelná.

Pro konstrukci měřicího vozu byla tedy vybrána platforma ATmega s Bluetooth
modulem HC-05.

\section{Mechanika}

Vůz se bude po mechanické skládat z několika základních částí.

\begin{enumerate}
\item Perforované kolo připevněné na nápravě.
\item Optozávora připevněná k senzoru.
\item Baterie.
\item Deska plošných spojů s hlavní elektronikou.
\item Komunikační modul HC-05.
\end{enumerate}

Jako vůz byl použit běžně dostupný vůz měřítka TT typu Nd\ref{vuz-nd}. Na
jedné straně bylo extrahováno spřáhlo a místo něj byla k nápravě umístěna
optozávora. Optozávora je na vlastní miniaturní DPS, k vozu je DPS mechanicky
připevněna lepidlem. Vůz Nd je otevřený vůz, autor si (minimálně v prototypu)
neklade cíl měřicí elektroniku modelově maskovat. Na plošině vozu je umístěna
baterie, na ní je řídicí elektronika a na ní je komunikační modul. Celou
situaci přehledně zobrazuje nákres \ref{vuz-nakres}

TODO: vuz-nakres

Rozměr plošiny vozu je $20\times80\ mm$. což umožnilo instalaci LiPol baterie
o kapacitě $500 mAh$ (napětí $3.7\ V$). Teoretická výdrž vozu je tedy cca $x$
hodin. Je však nutno poznamenat, že toto číslo je skutečně pouze teoretické,
spotřeba zařízení se výrazně liší ve spárovaném a nespárovaném stavu, navíc
níže zjistíme, že baterii nevybíjíme až na její hranice. Předestřeme však již
nyní naměřenou hodnotu, tj. že spárovaný vůz vydržel reálně komunikovat
$y$ hodin. To je velice uspokojivé číslo.

\section{Elektronika}

Základní požadavky na elektroniku měřicího vozu jsou:

\begin{enumerate}
\item Umět rychle a přesně vyhodnocovat data z optického seznoru.
\item Umět komunikovat s modulem HC-05.
\item Umožnit uživateli zařízení zapnout a vypnout.
\item Umět měřit napětí na baterce.
\item Umět automaticky vypnout zařízení v případě vybití baterie.
\item Zobrazovat stav vozu LE diodami.
\item Umožnit nabíjet baterii.
\end{enumerate}

Všechny požadavky na elektroniku jsou víceméně intuitivní, jen u posledního
doplňme, že jeho motivací je to, aby uživatel nemusel odpojovat baterii od
elektroniky a nabíjet jí externě. Tím se výrazně sníží riziko omylného
přepólování, nesprávného zacházení s baterií ze strany uživatele a mechanického
poškození vozu vlivem časté manipulace s kabely a konektory.

Schéma desky plošných spojů je k nahlédnutí v příloze \ref{wsm-scheme}. Nyní
krátce popišme jednotlivé prvky schématu a jak přispívají k naplnění požadavků.

Největší součástkou na desce je procesor ATmega328p. Tento konkrétní typ byl
vybrán, protože je poměrně moderní a je dnes montován do vývojových desek
Arduino, tudíž je jeho cena díky vysokému počtu produkovaných kusů malá.
Tento procesor se stará o všechnu logiku na desce. Signál z optozávory je
připojen na pin ICP, který je speciálně určen k tomu, aby přesně a rychle měřil
periodu vstupního signálu. Procesor dále měří napětí na baterce, které je ale
nejprve nutné srazit napěťovým děličem. Při vývoji hardwaru byla v jednu chvíli
překážkou přesnost napěťové reference, nakonec ale byla zvolena jako napěťová
reference přímo napájecí napětí procesoru, které by mělo kolísat nejvýše $\pm 2\ \%$
\ref{ldo-datasheet}.

Součástí schématu je dále nabíjecí obvod \textit{MCP73831}, který byl vybrán
zejména pro jeho velice malé rozměry.

Na tímto obvodem je ochrana proti přepólování (tranzistor \textit{T1A}) a dále
zapínací a vypínací obvod celé elektroniky. Tento obvod je inspirován projektem
RB3201\ref{rb3201} od kolegů z Robotiky Brno\ref{roboticsbrno}.

Napětí pro procesor je vytvářeno obvodem LDO\footnote{Low-dropout regulator}
(obvod \textit{IC3}), procesor pracuje na napětí $3.3\ V$. Tato konstrukce
s sebou nese zásadní omezení pro čerpání baterky: napětí na baterce nikdy nesmí
klesnout pod $3.5\ V$. Tato hodnota zahrnuje i korekce vypočítané z nepřesnosti
výstupu LDO, jedná se tedy o \textit{napětí, které měří procesor}. Jakmile
procesor na baterce změří méně než $3.5\ V$, musí se odpojit. Jinak hrozí
pokles referenčního napětí na AD převodníku, který měří napětí na baterce.

DPS dále obsahuje konektory pro periferie, testovací konektory, testovací
plošky a LE diody. Autor věří, že tyto komponenty nepotřebují žádný další
komentář.

Kompletní materiály k elektronice jsou k dispozici pod licencí CC BY-SA 4.0 na
serveru GitHub\ref{wsm-pcb}.

\section{Firmware}

TODO

---

TODO:
 * shrnout vlatnosti, které měly vliv na konstrukci
 * naměřit přesné spotřeby a odhady výdrže
 * podmínky na výdrž
