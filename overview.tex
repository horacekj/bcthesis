\documentclass[12pt,a4paper]{article}
\usepackage[unicode,colorlinks=true]{hyperref}
\usepackage[czech]{babel}
\usepackage[utf8]{inputenc}
\usepackage[T1]{fontenc}
\usepackage{lmodern}
\usepackage{graphicx}
\textwidth 16cm \textheight 24.6cm
\topmargin -1.3cm 
\oddsidemargin 0cm
\begin{document}
\pagestyle{empty}
\noindent

% Zatím kašlu na sázení v celé šabloně, nejdřív text, pak šablona.

\section{Přehledová kapitola}

V této kapitole budou představeny existující řešení problému měření rychlosti
modelového hnacího vozidla doplněné o komentáře vhodnosti jednotlivých řešení
pro problém řešený touto prací. Dále budou představeny dostupné technologie
senzorů, které jsou pro problém měření rychlosti vhodné.

\subsection{Statické bodové detektory}

Jednou z nejjednodušších technologií pro detekci rychlosti pohybujícího se
objektu jsou, zjednodušeně řečeno, 2 senzory detekující průchod daným bodem
a stopky. V naší aplikaci na modelovém kolejišti si lze představit 2
(např. optické) senzory umístěné v určité vzdálenosti od sebe, skutečnou
rychlost pohybujícího se objektu lze pak vypočíst jednoduchým podělením
vzdálenosti senzorů a času, po který se hnací vozidlo pohybovalo mezi senzory.

Měření vychází z předpokladu, že se hnací vozidlo pohybuje konstantní
rychlostí (podobně, jako řada dalších rozebíraných metod měření), tento
předpoklad je však zejména díky kontrole nad ovládáním vozidla a díky BEMF
[TODO] možné bez problému splnit.

Konkrétní instancí tohoto způsobu řešení na například Model Railroad Speedometer
Accutrack II [http://www.sprog.us.com/speedo.html].

Nespornými výhodami tohoto přístupu jsou jednoduchost a absence pohyblivých
prvků, která vede k delší životnosti a dlouhodobé spolehlivosti měřiče. Bohužel,
zásadní nevýhodou tohoto řešení je nutnost měřit rychlost pouze ne vybraném
úseku tratě. V požadavcích na řešení vyvíjené v této práci sice bylo specifikováno,
že kalibrace bude probíhat na uzavřeném okruhu, avšak představme si, že pro
jedno změření rychlosti by bylo třeba buď projet celý okruh, nebo se za senzorem
zastavit a obrátit směr. Pomineme-li fakt, že motory často dávají při stejné
střídě napájecího signálu mírně rozdílné otáčky v závislosti na směru, ve
kterém se pohybují, je i tak čas nutný na celou kalibraci nesrovnatelně vyšší
než u senzoru, který je schopen odečítat rychlost průběžně, nezávisle na
poloze vozidla.

Uvědomme si ale na závěr, že jedna z klíčových nevýhod tohoto senzoru je v
určitém slova smyslu i jeho výhodou. Totiž fakt, že měření probíhá jen na pevně
definovaném úseku koleje s sebou nese tu výhodu, že vozidlo měříme vždy ve
stejných podmínkách. Nemusíme tedy například řešit, jestli vozidlo zrovna jede
v oblouku a je zpomalováno odstředivými silami, protože senzor prostě umístíme
na rovnou trať.

\subsection{Statické úsekové detektory}

Podobn

\end{document}
