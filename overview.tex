\documentclass[12pt,a4paper]{article}
\usepackage[unicode,colorlinks=true]{hyperref}
\usepackage[czech]{babel}
\usepackage[utf8]{inputenc}
\usepackage[T1]{fontenc}
\usepackage{lmodern}
\usepackage{graphicx}
\textwidth 16cm \textheight 24.6cm
\topmargin -1.3cm 
\oddsidemargin 0cm

\usepackage[ backend=biber
           , style=numeric
           , sortlocale=en_US
           , bibencoding=UTF8
           , maxcitenames=3
           , maxbibnames=100
]{biblatex}
\addbibresource{bibliography.bib}

\begin{document}
\pagestyle{empty}
\noindent

% Zatím kašlu na sázení v celé šabloně, nejdřív text, pak šablona.


\section{Přehledová kapitola}

V této kapitole budou představeny existující řešení problému měření rychlosti
modelového hnacího vozidla doplněné o komentáře vhodnosti jednotlivých řešení
pro problém řešený touto prací. Dále budou představeny dostupné technologie
senzorů, které jsou pro problém měření rychlosti vhodné.

\subsection{Statické bodové detektory}

Jednou z nejjednodušších technologií pro detekci rychlosti pohybujícího se
objektu jsou, zjednodušeně řečeno, 2 senzory detekující průchod daným bodem
a stopky. V naší aplikaci na modelovém kolejišti si lze představit 2
(např. optické) senzory umístěné v určité vzdálenosti od sebe, skutečnou
rychlost pohybujícího se objektu lze pak vypočíst jednoduchým podělením
vzdálenosti senzorů a času, po který se hnací vozidlo pohybovalo mezi senzory.

Měření vychází z předpokladu, že se hnací vozidlo pohybuje konstantní
rychlostí (podobně, jako řada dalších rozebíraných metod měření), tento
předpoklad je však zejména díky kontrole nad ovládáním vozidla a díky BEMF
možné bez problému splnit.

Konkrétní instancí tohoto způsobu řešení na například Model Railroad Speedometer
Accutrack II \cite{accutrackII}.

Nespornými výhodami tohoto přístupu jsou jednoduchost a absence pohyblivých
prvků, která vede k delší životnosti a dlouhodobé spolehlivosti měřiče. Bohužel,
zásadní nevýhodou tohoto řešení je nutnost měřit rychlost pouze ne vybraném
úseku tratě. V požadavcích na řešení vyvíjené v této práci sice bylo specifikováno,
že kalibrace bude probíhat na uzavřeném okruhu, avšak představme si, že pro
jedno změření rychlosti by bylo třeba buď projet celý okruh, nebo se za senzorem
zastavit a obrátit směr. Pomineme-li fakt, že motory často dávají při stejné
střídě napájecího signálu mírně rozdílné otáčky v závislosti na směru, ve
kterém se pohybují, je i tak čas nutný na celou kalibraci nesrovnatelně vyšší
než u senzoru, který je schopen odečítat rychlost průběžně, nezávisle na
poloze vozidla.

Uvědomme si ale na závěr, že jedna z klíčových nevýhod tohoto senzoru je v
určitém slova smyslu i jeho výhodou. Totiž fakt, že měření probíhá jen na pevně
definovaném úseku koleje s sebou nese tu výhodu, že vozidlo měříme vždy ve
stejných podmínkách. Nemusíme tedy například řešit, jestli vozidlo zrovna jede
v oblouku a je zpomalováno odstředivými silami, protože senzor prostě umístíme
na rovnou trať.

\subsection{Statické úsekové detektory}

Podobný přístup, jako u statických bodových detektorů, nabízejí statické úsekové
detektory. Rozdíl oproti senzorům popsaným v předchozí kapitole je v tom, že
tyto senzory neměří projetí vozidla daným bodem, ale přítomnost vozidla na daném
úseku tratě.

Typickým zástupce této technologie je proudový detektor obsazení, konkrétně
například BD20 \cite{bd20}.

Tyto technologie jsou hojně využívané na modelových kolejištích k detekci
volnosti daného úseku a tudíž k umožnění bezkolizního pohybu vozidla po trati.
Obdobné systémy se využívají na skutečné železnici, avšak tam jsou známy pod
názvem \textit{kolejové obvody}.

Konkrétní způsob měření přehledně popisují například tvůrci SW JMRI
\cite{jmri:speedometer}.
Kalibrační kolej je rozdělena na několik částí (optimálně 3 části), kalibrační
SW pak postupně projíždí těmito částmi a počítá čas, za který vozidlo projede
konkrétní část. U této části má od uživatele zadanou její délku, z čehož pak
vypočte rychlost vozidla.

Výhody a nevýhody tohoto způsobu měření jsou víceméně stejné s výhodami a
nevýhodami statických bodových detektorů. Malou výhodou úsekových detektorů by
mohlo být, že kolejiště často bývají vybavena úsekovými detektory pro účely
zapezpečovacího zařízení, takže není nutný nákup a instalace dalších senzorů.

\subsection{Podvalníkové detektory}

Zajímavým způsobem měření rychlosti prezentovaným v \cite{bachrus}
je možnost měřit rychlosti hnacího vozidla bez toho, aby se hnací vozidlo
skutečně pohybovalo po trati.

Tuto metodu si lze představit podobně, jako měření rychlostí aut v autoservisu.
Lokomotiva je položena na válce, které se pod ní mohou volně pohybovat. Lokomotiva
svým motorem roztáčí válce, jejichž rychlost je následně měřena.

Zajímavou výhodou této technologie je použitelnost prakticky pro jakékoliv
modelové měřítko, nezávislost na pozici vozidla na trati a v neposlední řadě
zmizení nutnosti pořizovat kalibrační okruh. Celý proces kalibrace tak může
probíhat na velice malém prostoru.

Je rozhodně nutné uznat, že toto řešení je svým důvtipem minimálně oceněníhodné,
avšak pro potřeby této práce skrývá jednu zásadní nevýhodu. Totiž nemožnost
dokalibrovávat vozidlo za reálného provozu bez nutnosti uvolnění vozidla
z reálného provozu na trati.

Stojí za to také poznamenat, že řešení, které by umožňovalo měřit rychlost vozidla
na kolejišti by s sebou neslo tu výhodu, že je lze použít k měření celkové
ujeté vzdálenosti hnacího vozidla za delší čas, což je údaj využitelný například
k výpočtu míry opotřebení vozidla.

\subsection{Kalibrační SW}

Druhou, zcela zásadní části této práce, je mimo tvorby senzoru rychlosti hnacího
vozidla také vytvoření programu, který zvládne celý proces kalibrace automaticky.
V tomto hledisku autor bohužel nemůže nabídnout srovnání s existujícími řešeními,
protože takové řešení prakticky neexistují. Je to z toho důvodu, že většina
dostupných SW pro řízení modelového kolejiště staví na jiném principu, než na
kterém staví řídicí SW hJOP \cite{hjop:web}, kvůli kterému je automatická kalibrace
řešena.

Většina obslužných programů totiž počítá s tím, že hnací vozidla na kolejišti
jsou různá a nesnaží se je jednotě kalibrovat. Místo toho si tyto SW před prvním
použitím lokomotivy nechají zadat její parametry nebo si je změří jednou
z technologií popsanými výše a u lokomotivy si tuto \textit{kalibrační tabulku}
udržují. Tyto programy tak kompenzují rozdílné vlastnosti vozidel přímo povelováním
různých vozidel různými rychlostmi.

Kocepce sw hJOP je však jiná. Bylo rozhodnutím autora, že je vhodnější mít
zkralibrovaná vozidla na úrovni dekodérů, než si v SW ukládat pro každé hnací
vozidlo speciální \textit{kalibrační tabulku}.

\subsection{TODO}

\textit{Ve skutečné bakalářce tady bude ještě přehled různých senzorů, ale
kvůli rozsahu do předmětu VB000 sem už další text neuvádím.}

\section*{Bibliografie}
%\addcontentsline{toc}{chapter}{Bibliography}
%\markboth{}{} % avoid headers from last chapter in bibliography
\printbibliography[heading=none]

\end{document}
