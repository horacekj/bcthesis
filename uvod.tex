V~následujících několika stranách autor rozebere možnosti řešení problému
automatické kalibrace modelového žlezničního hnacího vozidla, ukáže, proč
jsou současná řešení pro problém řešený v~této páci nevhodná a nakonec
navrhne (1) hardware, (2) firmware a (3) software, kterým se pokusí problém
řešit. Autor demonstruje výstupy této práce v~praxi a ukáže její reálný
přínos.

Tato práce se tématicky pohybuje na pomezí práce typu \textit{proof of concept}
a práce \textit{aplikační}. Jejím primárním cílem je vytvořit reálné řešení
reálného problému v~oblasti automatizace, konkrétně v~oblasti řízení dopravy na
modelovém kolejišti. Jakkoliv se může zdát, že toto téma se nachází velice
blízko hardwaru, primárním cílem této práce je navrhnout kvalitní softwarové
řešení.

Pro přesnou formulací problému automatické kalibrace železničního vozidla je
nutné pochopit kontext této práce, o kterém pojednává následující kapitola.

\section{Modelová kolejiště}

Modelové kolejiště je model reálné železniční tratě, typicky zmenšený
v~některém ze standardních měřítek, např. v~měřítku \uv{TT} (1:120). Vedle
typického \uv{hobby} využití kolejiště často slouží jako dopravní trenažéry pro
výuku zabezpečovacích zařízení železnice, či dalších aspektů využívaných na
skutečné železnici. Nejen v~těchto případech vzniká potřeba kolejiště ovládat
\textit{zabezpečovacím zařízením}. Typickým zástupcem současného
zabezpečovacího zařízení na skuteční železnici je \textit{elektronické
stavědlo} spolu s~grafickou nadstavbou \textit{JOP}, která umožňuje řízení
stavědla přes počítač.

Obdobný systém je využíván také v~\textit{Klubu modelářů železnic Brno I},
který je základní motivací k~vypracování této práce. V~tomto klubu je nasazen
systém řízení kolejišť zvaný \textit{hJOP} \cite{hjop:web}, který je klonem JOP
upraveným pro potřeby řízení modelu. Jeho primárním cílem je povelovat jízdu
vlaku a omezením lidského faktoru předejít možným kolizím.

\section{Současné trendy v~řízení kolejiště}

K~řízení modelových kolejišť je v~současné době napříč zeměmi využívána celá
řada dostupných SW. Typickými příklady jsou například \textit{JMRI}
\cite{jmri:web}, \textit{TrainController} \cite{trancontroller:web},
\textit{RailCo} \cite{railco:web} nebo český SW \textit{modelJOP}
\cite{modeljop:web}.

Výše zmíněné aplikace komunikují s~hardwarovými prvky kolejiště.
Nejdůležitějším z~těchto prvků je digitální centrála DCC. Digitální centrála je
obecný pojem, konkrétními instancemi od jednotlivých výrobců jsou pak např.
\textit{Z21} \cite{z21:web}, \textit{NanoX} \cite{nanox:web} nebo třeba
\textit{Intellibox} \cite{intellibox:web}. Centrála typicky komunikuje
s~počítačem pomocí sběrnice \textit{RS232 over USB}. Jejím hlavním úkolem je
vytvářet \textit{digitální signál DCC}, který moduluje do kolejí.

Lokomotiva stojící na kolejích tento signál dekóduje pomocí speciální
součástky -- tzv. \textit{dekodéru} -- a podle přijatých dat poveluje periferie
lokomotivy: motor, světla, nebo třeba zvuk vydávaný vestavěným reproduktorem.
Signál v~kolejích je zároveň využíván jako výkonový (např. pro napájení
motoru). Dekodér v~podstatě obsahuje jen jednoduchý mikrokontrolér a několik
výkonových prvků. Jeho typická velikost je v~řádu malých jednotek $cm^2$, takže
je skutečně malý.

Každý dekodér má svoji adresu v rozsahu $1-9999$. Počítač pak může vydat příkaz
pro řízení lokomotivy, např.: \textit{\uv{lokomotivo s~adresou 562, jeď směrem
vpřed rychlostí 15!}}

\section{Synchronizace rychlosti vozidel}

Rychlost lokomotivy typický bývá přirozené číslo v~rozsahu $0-28$. Tato hodnota
se označuje jako tzv. \textit{jízdní stupeň}.

Protože každá lokomotiva obsahuje jiné mechanické prvky (převody, motor),
odpovídá stejnému jízdnímu stupni u~dvou různých hnacích vozidel různá
rychlost. Ukazuje se, že při řízení kolejiště je vhodné spárovat jízdní stupeň
s~reálnou rychlostí vozidla. Je to především proto, abychom dosáhli modelové
věrnosti a vozidlo se po přepočtu pohybovalo \textit{modelovou rychlostí}. Je
také nutné dbát na provozní parametry kolejiště, které umožňují průjezd
některými části kolejiště jen omezenou rychlostí. Při překročení této rychlosti
by pak mohlo dojít k~vykolejení, což je samozřejmě nežádoucí.

Dalším důvodem pro spárování rychlosti lokomotivy s~jízdním stupněm je
vytvoření předvídatelného prostředí pro obsluhu. Obsluha má typicky k~dispozici
ovladač pro řízení rychlosti jízdy vozidla v~ručním režimu. Autor této práce
považuje za velice užitečné, aby platilo, že otočení trimru ovladače o~daný
úhel způsobí u~všech vozidel stejnou rychlost.

Třetím důvodem k synchronizaci rychlostí vozidel je umožnění přípřeží a postrků
-- tj.  situací, kdy v~jednom vlaku jede více lokomotiv. Tyto lokomotivy se
musí pohybovat stejnou rychlostí, jinak hrozí vykolejení vlaku.

\subsection{Současné řešení problému}

Současné řídicí programy problém synchronizace rychlostí typicky řeší tak, že
si u~každého hnacího vozidla ukládají tabulku \textit{jízdní stupeň: skutečná
rychlost}. Když tyto programy chtějí, aby 2 lokomotivy jely stejnou rychlostí,
provedou vyhledání příslušných jízdních stupňů a do centrály (v~obecném
případě) odešlou 2 různé jízdní stupně.

Provozu lokomotivy na kolejišti tak předchází buď

\begin{itemize}
	\item ruční zadání této tabulky, nebo
	\item automatické měření rychlosti lokomotivy.
\end{itemize}

Je však nutné podotknout, že toto řešení neuspokojuje naše požadavky pro
ruční řízení jízdy hnacího vozidla, protože tabulka rychlostí je uložena
pouze v~řídícím SW, s~kterým nejsou ovladače typicky propojeny, takže o~tabulce
vůbec netuší.

\subsection{Řešení problému v~hJOP}

Alternativním řešením problému synchronizace rychlostí je způsob, který
v~současné době využívá SW hJOP. Tato práce cílí právě na tento alternativní
způsob řešení kalibrace. Jeho výhoda je v~tom, že řeší synchronizaci
rychlostí i~pro ruční ovladače.

V~každém lokomotivním dekodéru je možno pro každý z~28 jízdních stupňů nastavit
konkrétní výkon motoru. SW hJOP staví na předpokladu, že uživatel provede před
provozem hnacího vozidla tzv. \textit{kalibraci}, tj. přiřazení výkonu motoru
každému rychlostnímu stupni každého vozidla tak, aby bylo splněno, že konkrétní
jízdní stupeň odpovídá u~všech vozidel konkrétní rychlosti.

Příklad: u~všech lokomotiv platí, že jízdní stupeň $15$ odpovídá reálné
rychlosti $40\ km/h$.

Dalšími výhodami tohoto řešení jsou:

\begin{enumerate}
	\item odpadnutí nutnosti udržovat si u~každého vozidla v~SW kalibrační
	tabulku
	\item a instantní přenos tabulky mezi kolejišti, když je přenesena
	lokomotiva. To proto, že tabulka je jednoduše fyzicky uložená v~lokomotivě a
	tudíž logicky cestuje s ní.
\end{enumerate}

Nevýhodou tohoto přístupu je, že je nutné provést netriviální proces kalibrace.
V~současné době tento proces zahrnuje ježdění s~lokomotivou na měřícím okruhu,
měření její rychlosti a ruční nastavování kalibrační tabulky.

Cílem této práce je tento proces automatizovat.

\section{Cíl práce}

Cílem této práce je automatizovat proces kalibrace lokomotivy.

Nejprve je nutné navrhnout metodu pro měření rychlosti hnacího vozidla. Jakmile
bude zabezpečen přenos dat o~rychlosti hnacího vozidla do řídícího PC, je nutné
navrhnout SW, který na nákladě změřené rychlosti provede nastavení kalibrační
tabulky hnacího vozidla.
