V následujících několika stranách rozebere autor možnosti řešení problému
automatické kalibrace modelového žlezničního hnacího vozidla, ukáže, proč
jsou současná řešení pro problém formulovaný v této páci nevhodná a nakonec
navrhne (1) hardware, (2) firmware a (3) software, který se pokusí problém
řešit. Autor demonstruje výstupy této práce v praxi a ukáže její reálný
přínos.

Tato práce se tématicky pohybuje na pomezí práce typu \textit{proof of concept}
a práce \textit{aplikační}. Jejím primárním cílem je vytvořit reálné řešení
reálného problému v oblasti automatizace, konkrétně řízení dopravy na modelovém
kolejišti. Jakkoliv se může zdát, že toto téma se nachází velice blízko
hardwaru, primárním cílem této práce je navrhnout kvalitní softwarové řešení.

Pro přesnou formulací problému automatické kalibrace železničního vozidla je
nutné pochopit kontext této práce, o kterém pojednává následující odstavec.

\section{Kontext}

Modelové kolejiště je model reálné železniční tratě, tyicky zmenšený v některém
ze standardních měřítek, např. v měříku \uv{TT} 1:120. Vedle typického \uv{hobby}
využití kolejiště je dobré mít na paměti, že takováto kolejiště často slouží
třeba jako dopravní trenažéry pro výuku zabezpečovacího zařízení železnice,
či dalších aspektů využívaných na skutenčé železnici. Nejen v těchto případech
vzniká potřeba kolejiště ovládat nějakým \textit{zabezpečovacím zařízením}.
Typickým zástupcem současného zabezpečovacího zařízení je
\textit{elektronické stavědlo} spolu s grafickou nadstavbou \textit{JOP}.

Tento systém je využíván také v \textit{Klubu modelářů železnic Brno I}, ke
kterému má autor práce blízko. V něm je nasazen systém řízení zvaný
\textit{hJOP}, který je klonem skutečného zabezpečovacího zařízení upraveného
pro potřeby řízení modelu. Jeho primárním cílem ve výše zmíněném klubu je
kontrolovat jízdu vlaku a omezením lidského faktoru předejít možným škodám.

\section{Řízení kolejiště}

K řízení modelového kolejiště je v současné době napříč země využívána celá
řada dostupných SW. Typickými příklady jsou například JMRI \cite{},
TrainController \cite{}, RailCo \cite{} nebo česky SW modelJOP \cite{}.

Tyto programy komunikují s hardwarovými prvky v kolejišti. V dnešní době je
nejdůležitějším z těchto prvků digitální centrála DCC. Digitální centrála je
obecný pojem, konkrétními instancemi od jednotlivých výrobců jsou pak např.
Z21 \cite{}, NanoX \cite{} nebo třeba Intellibox \cite{}. Digitální centrála
typicky komunikuje s počítačem pomocí \textit{RS232 over USB}. Jejím hlavním
úkolem je vytvářet \textit{digitální signál DCC}, který posílá do kolejí.

Lokomotiva stojící na kolejích tento signál dekóduje pomocí speciální součástky,
tzv. \textit{dekodéru} a podle přijatých dat poveluje periferie lokomotivy, jako
jsou třeba motor, světla, nebo zvuk vydávaný vestavěným reproduktorem. Signál
v kolejích je zároveň využíván jako výkonový (např. pro napájení motoru).
Dekodér není žádná složitá součástka, v podstatě obsahuje jednoduchý
mikrokontrolér a několik výkonových prvků. Jeho typická velikost je v řádu
malých jednotek $cm^2$, takže je skutečně malý.

Každý dekodér má svoji adresu v rozsahu $1-9999$. Počítač pak může vydat příkaz
pro řízení lokomotivy, např. \textit{\uv{Lokomotivo s adresou 562, jeď směrem
vpřed rychlostí 15}}.

\section{Kalibrace vozidel}

\subsection{Nutnost kalibrace}

Rozsah rychlostí lokomotivy typický bývá v rozsahu $0-28$. Tato hodnota se
označuje jako tzv. \textit{jízdní stupeň}.

Protože každá lokomotiva má jiné převody a jiný motor, odpovídá stejnému
jízdnímu stupni u dvou různých hnacích vozidel různá skutečná rychlost těchto
hnacích vozidel. Ukazuje se, že při řízení kolejiště je nutné spárovat
konkrétní jízdní stupeň s reálnou rychlostí konkrétního vozidla. Je to
především proto, abychom dosáhli modelové věrnosti a vozidlo se po přepočtu
pohybovalo \textit{modelovou rychlostí}. Je také nutné dbát na provozní
parametry kolejiště, které jednoduše umožňují průjezd některými části kolejiště
jen omezenou rychlosti. Při překročení této rychlosti by pak mohlo dojít k
vykolejení, což je nežádoucí.

Další, neméně důležitým důvodem pro spárování skutečné rychlosti s jízdním
stupněm je vytvoření předvídatelného prostředí pro obsluhu. Obsluha má typicky
k dispozici ovladač pro řízení rychlosti jízdy vozidla v ručním režimu. Autor
této práce považuje za velice užitečné, aby platilo, že otočení trimru ovladače
o daný úhel způsobí u dvou různých vozidel stejnou rychlost.

\subsection{Současné řešení problému kalibrace}


