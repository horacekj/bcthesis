V následujících několika stranách rozebere autor možnosti řešení problému
automatické kalibrace modelového žlezničního hnacího vozidla, ukáže, proč
jsou současná řešení pro problém formulovaný v této páci nevhodná a nakonec
navrhne (1) hardware, (2) firmware a (3) software, který se pokusí problém
řešit. Autor demonstruje výstupy této práce v praxi a ukáže její reálný
přínos.

Tato práce se tématicky pohybuje na pomezí práce typu \textit{proof of concept}
a práce \textit{aplikační}. Jejím primárním cílem je vytvořit reálné řešení
reálného problému v oblasti automatizace, konkrétně řízení dopravy na modelovém
kolejišti. Jakkoliv se může zdát, že toto téma se nachází velice blízko
hardwaru, primárním cílem této práce je navrhnout kvalitní softwarové řešení.

\section{Kontext }
