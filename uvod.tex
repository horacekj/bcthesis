Současná modelová kolejiště již dávno nejsou jen hračkou malých dětí
a jejich rodičů. Jsou to plnohodnotné modely, na kterých se dispečeři skutečné
železnice učí řídit dopravu a~které slouží pro výuků oborů automatizace.
Kolejiště jsou vybavena robustní a~spolehlivou zabezpečovací technikou
podobnou té, která se v~současnosti používá na skutečné železnici.

S~růstem rozměrů kolejišť a~cen modelů vzniká nutnost procesy řízení
automatizovat, předejít chybám způsobeným lidským faktorem a~řízení kolejiště
obecně zefektivnit. Cílem této práce je pomocí automatizace procesu kalibrace
modelového železničního vozidla přispět právě ke zvýšení bezpečnosti
a~spolehlivosti provozu.

Autor rozebere možnosti řešení problému automatické kalibrace, ukáže, proč jsou
současná řešení nevhodná a~nakonec navrhne (1) hardware, (2) firmware a~(3)
software, kterými uvedený problém řeší. Autor demonstruje výstupy této práce
v~praxi a~ukáže její reálný přínos.

Tuto práci je tematicky možné zařadit jednak do kategorie \textit{proof of
concept}, druhak však i~do kategorie práce \textit{aplikační}. Jejím primárním
cílem totiž je vytvořit reálné řešení reálného problému v~oblasti automatizace,
který doteď ještě nikdo neřešil. Konkrétně se práce zaměřuje na automatizaci
řízení dopravy na modelovém kolejišti. Cílem této práce je navrhnout kvalitní
řešení od hardwaru, přes firmware až po software.

Pro přesnou formulaci problému automatické kalibrace železničního vozidla je
nutné pochopit kontext této práce, o~kterém pojednává tato kapitola.

\section{Modelová kolejiště}
\label{sec:mod-kol}

Modelové kolejiště je model železniční tratě, typicky zmenšený
v~některém ze standardních měřítek, např. v~měřítku \uv{TT} (1:120) či \uv{H0}
(1:87). Vedle typického \uv{hobby} využití kolejiště často slouží jako dopravní
trenažéry pro výuku zabezpečovacích zařízení železnice či dalších aspektů
využívaných na skutečné železnici. Nejen v~těchto případech vzniká potřeba
kolejiště ovládat \textit{zabezpečovacím zařízením}. Typickým příkladem
současného zabezpečovacího zařízení na skutečné železnici je
\textit{elektronické stavědlo} spolu s~grafickou nadstavbou
\textit{JOP} (\textit{Jednotné obslužné pracoviště}), která umožňuje řízení
stavědla přes počítač.

Obdobný systém je využíván také v~\textit{Klubu modelářů železnic Brno I}.
V~tomto klubu je nasazen systém řízení kolejišť zvaný \textit{hJOP}
\cite{hjop:web}, který je klonem JOP upraveným pro potřeby řízení modelu. Jeho
primárním cílem je řídit jízdu vlaku a~omezením lidského faktoru předejít
možným kolizím.

\section{Současné trendy v~řízení kolejiště}
\label{sec:trendy}

K~řízení modelových kolejišť je v~současné době napříč zeměmi využívána celá
řada dostupných SW. Typickými příklady jsou například \textit{JMRI}
\cite{jmri:web}, \textit{TrainController} \cite{traincontroller:web}
nebo český SW \textit{modelJOP} \cite{modeljop:web}.

Výše zmíněné aplikace potřebují komunikovat s~hardwarovými prvky kolejiště.
Nejdůležitějším z~těchto prvků je digitální centrála DCC. Digitální centrála je
obecný pojem, konkrétními instancemi od jednotlivých výrobců jsou pak např.
\textit{Z21} \cite{z21:web}, \textit{NanoX} \cite{nanox:web} nebo třeba
\textit{Intellibox} \cite{intellibox:web}. Centrála nejčastěji komunikuje
s~počítačem pomocí sběrnice \textit{RS232 over USB}. Jejím hlavním úkolem je
vytvářet \textit{digitální signál DCC} \cite{nmra:dcc:ele}, který moduluje do
kolejí.

Lokomotiva stojící na kolejích tento signál dekóduje pomocí spe\-ciál\-ní
součástky -- tzv. \textit{dekodéru DCC} -- a~podle přijatých dat poveluje
periferie lokomotivy: motor, světla, nebo třeba zvuk vydávaný vestavěným
reproduktorem. Modulovaný výkonový signál v~kolejích je zároveň využíván jako
(1) nosič informace pro dekodéry a~(2) po usměrnění jako napájecí napětí pro
periferie, včetně výkonově nejnáročnější periferie -- motoru. Dekodér
v~podstatě obsahuje jen jednoduchý mikrokontrolér a~několik výkonových prvků.
Jeho typická velikost je v~řádu malých jednotek cm$^2$, takže je skutečně malý.

Každý dekodér má svoji adresu v~rozsahu $1$--$9999$. Počítač pak může vydat příkaz
pro řízení lokomotivy, např.: \textit{\uv{lokomotivo s~adresou 562, jeď směrem
vpřed rychlostí 15!}}

Každý dekodér si ukládá svá konfigurační data do volatilní paměti formou tzv.
\textit{CV} (\textit{Configuration Value}). \textit{CV} je jedna konfigurační
jednotka s~rozsahem hodnot 1~byte ($0$--$255$). Každé \textit{CV} má svoje číslo
(typicky $1$--$1023$). Výrobce dekodéru (potažmo standardizační organizace) pak
definuje význam jednotlivých CV. Například CV $\#17$ a~$\#18$ jsou vyhrazeny
k~uložení adresy dekodéru \cite{zimo:cvs}.

\section{Synchronizace rychlosti vozidel}
\label{sec:sync-rych}

Rychlost lokomotivy typicky bývá přirozené číslo v~rozsahu $0$--$28$. Tato
hodnota se označuje jako tzv. \textit{jízdní stupeň}.

Protože každá lokomotiva obsahuje jiné mechanické prvky (převody, motor),
odpovídá stejnému jízdnímu stupni u~dvou různých nezkalibrovaných hnacích
vozidel různá rychlost. Ukazuje se, že při řízení kolejiště je vhodné, aby (1)
stejnému jízdnímu stupni u~všech hnacích vozidel na kolejišti odpovídala stejná
rychlost a~aby (2) byl jízdní stupeň pevně spárovaný s~reálnou rychlosti
vozidla přepočtenou dle měřítka.

Synchronizací rychlosti vozidel myslíme právě splnění těchto dvou vlastností
pro všechny lokomotivy na kolejišti. Proces synchronizace rychlostí je součásti
procesu \textit{kalibrace lokomotivy}.

Motivací pro tyto kroky je především to, abychom dosáhli modelové věrnosti
a~aby se vozidlo po přepočtu pohybovalo \textit{modelovou rychlostí}. Je také
nutné dbát na provozní parametry kolejiště, které umožňují průjezd některými
částmi kolejiště jen omezenou rychlostí. Při překročení této rychlosti by pak
mohlo dojít k~vykolejení, což je samozřejmě nežádoucí.

Dalším důvodem pro synchronizaci je vytvoření předvídatelného prostředí pro
obsluhu. Obsluha má často k~dispozici ovladač pro řízení rychlosti jízdy
vozidla v~ručním režimu. Autor této práce považuje za velice užitečné, aby
platilo, že otočení trimru ovladače o~daný úhel způsobí u~všech vozidel pohyb
stejnou rychlostí (úhel otočení trimru ovladače odpovídá jízdnímu stupni).

Třetím důvodem k~synchronizaci rychlostí vozidel je umožnění přípřeží a~postrků
-- tj.  situací, kdy v~jednom vlaku jede více lokomotiv. Tyto lokomotivy se
musí pohybovat stejnou rychlostí, jinak hrozí vykolejení vlaku.

\subsection{Současné řešení problému}

Současné řídicí programy problém synchronizace rychlostí typicky řeší tak, že
si u~každého hnacího vozidla ukládají tabulku \textit{jízdní stupeň: rychlost}.
Když tyto programy chtějí, aby dvě lokomotivy jely stejnou rychlostí, vyhledají
příslušné jízdní stupně a~do centrály (v~obecném případě) odešlou pro každou
lokomotivu jiný jízdní stupeň.

Provozu lokomotivy na kolejišti tak předchází buď

\begin{compactitem}
	\item ruční zadání této tabulky, nebo
	\item automatické měření rychlosti lokomotivy.
\end{compactitem}

Je však nutné podotknout, že toto řešení neuspokojuje naše požadavky pro
ruční řízení jízdy hnacího vozidla, protože tabulka rychlostí je uložena
pouze v~řídicím SW, s~kterým nebývají ovladače propojeny, takže nemohou tabulku
využít. Stejnému úhlu otočení trimru ovladače tak zpravidla odpovídají různé
rychlosti u~různých lokomotiv, přípřeže a~postrky jsou prakticky nemožné.

\subsection{Řešení problému v~hJOP}

Alternativním řešením problému synchronizace rychlostí je způsob, který
v~současné době využívá SW hJOP. Tato práce cílí právě na onen alternativní
způsob řešení kalibrace. Jeho výhoda je v~tom, že řeší synchronizaci
rychlostí i~pro ruční ovladače.

V~každém lokomotivním dekodéru je možno pro každý z~28 jízdních stupňů nastavit
konkrétní otáčky motoru. SW hJOP předpokládá, že uživatel provede před
provozem hnacího vozidla tzv. \textit{kalibraci}, tj. přiřazení otáček motoru
každému rychlostnímu stupni každého vozidla tak, aby bylo splněno, že konkrétní
jízdní stupeň odpovídá u~všech vozidel stejné rychlosti.

Příklad: u~všech lokomotiv platí, že jízdní stupeň $15$ odpovídá rychlosti $40\
km/h$.

Dalšími výhodami tohoto řešení jsou:

\begin{compactenum}
	\item odpadnutí nutnosti udržovat si u~každého vozidla v~SW kalibrační
	tabulku a
	\item instantní přenos tabulky mezi kolejišti, když je přenesena
	lokomotiva (to proto, že tabulka je jednoduše fyzicky uložená v~lokomotivě,
	a tudíž logicky cestuje s~ní).
\end{compactenum}

Nevýhodou tohoto přístupu je, že je nutné provést netriviální proces kalibrace.
V~současné době tento proces zahrnuje ježdění s~lokomotivou na měřicím okruhu,
měření její rychlosti a~ruční nastavování kalibrační tabulky.

Cílem této práce je tento proces automatizovat.

\section{Synchronizace brzdných křivek vozidel}
\label{sec:sync-krivky}

Možná ještě fundamentálnějším problémem, než synchronizace rychlostí, je
synchronizace délky dojezdu z~určité fixní rychlosti napříč různými vozidly.
Lokomotivní dekodér má pro tento účel vyhrazeno CV $\#4$ \cite{zimo:cvs}, které
definuje dobu brzdění. Při nenulové hodnotě tohoto CV tak dekodér po obdržení
příkazu \uv{stůj} nezastaví ihned, ale postupně zpomaluje.

Tento parametr je naprosto klíčový, ovládací SW totiž typicky vyžaduje, aby dané
vozidlo zastavilo na konkrétní pozici (před návěstidlem, u~nástupiště, ...).
Současné ovládací aplikace problém synchronizace dojezdů řeší typicky tak, že si
dojezd změří a~dále s~ním kalkulují v~dalších výpočtech. Například program
\textit{TrainController} \cite{traincontroller:web} dokonce při brzdění
posílá dekodéru postupně povely \uv{jeď stupněm 12}, \uv{jeď stupněm 10},
\uv{jeď stupněm 5}, ..., takže se efekt brzdné křivky prakticky eliminuje.

Autor této práce považuje toto řešení za nešťastné, neboť
\begin{compactenum}
\item se zásadně zvyšuje datový tok mezi SW a~centrálou,
\item je třeba si u~každé lokomotivy udržovat další data,
\item po převzetí lokomotivy na ruční ovladač se každá lokomotiva chová jinak a
\item při jízdě přes ruční ovladač jsou typicky dojezdy tak malé, že lokomotiva
      brzdí nemodelově rychle.
\end{compactenum}

Proto byl při tvorbě SW hJOP zvolen alternativní přístup. Je striktně
definováno, že \textit{každá lokomotiva musí z~rychlosti 40 km/h zastavit na
$30$--$35\ cm$}.\footnote{Tato hodnota se liší pro jednotlivá měřítka,
$30$--$35\ cm$ je standard pro měřítko TT 1:120.} Do této vzdálenosti od
kýženého bodu zastavení jsou do kolejiště fyzicky montovány senzory. Po
projetí vozidla senzorem pošle hJOP povel \uv{lokomotivo s~adresou 1254, stůj!},
lokomotiva začne plynule brzdit a~do přesně dané vzdálenosti zastaví. A~to vše
modelově plynule.

Kalibrace dojezdů zajišťuje to, že hnací vozidlo nikdy nepřejede návěstidlo nebo
nenajede do špatně postavené výměny. Je proto ještě důležitější než kalibrace
rychlosti.\footnote{Kdyby uživatel zkalibroval lokomotivu tak, aby jezdila
hodně rychle, ale pořád dodržel $30$--$35cm$ brzdnou vzdálenost z~rychlostního
stupně 15 (ten odpovídá dle standardu KMŽ $40\ km/h$), lokomotiva nikdy nevjede,
kam nemá.}

Součástí procesu kalibrace by tedy měla být také synchronizace brzdných křivek.

\section{Cíl práce}
\label{sec:cil}

Cílem této práce je automatizovat proces kalibrace lokomotivy.

Nejprve je nutné navrhnout metodu pro měření rychlosti hnacího vozidla
a~zabezpečit přenos naměřených dat do řídicího PC. Poté je nutné navrhnout SW,
který na nákladě změřené rychlosti nastaví kalibrační tabulky hnacího vozidla.

Výstupem této práce je řešení, které umožní proces kalibrace především
automatizovat, což povede k~ušetření lidských kapacit. Dále umožní proces
kalibrace zrychlit, což přispěje k~minimalizaci režijního času spojeného
s~provozem kolejiště, a~zároveň proces kalibrace zpřesnit, což povede ke
zvýšení spolehlivosti provozu na kolejišti.
