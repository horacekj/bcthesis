Pro potřeby této práce byly vytvořeny 2 aplikace. První aplikace je
demonstrační aplikací k~měřicímu vozu, tato aplikace pouze zobrazuje naměřená
data. Druhá aplikace je o~mnoho komplexnější a řeší přímo problém automatické
kalibrace.

\section{Volba nástrojů}

Zásadní otázkou, kterou je třeba zodpovědět před programováním aplikace, je
jaký programovací jazyk a případně framework zvolit. Zde měl autor poměrně
jasný požadavek na programovací jazyk: jazyk musí mít poměrně silný typový
systém tak, aby spousta kontrol proběhla již během překladu. Důvodem pro tento
argument je fakt, že autor vyrábí aplikaci, které je tzv. \textit{kritická}.
V kontextu této práce to znamená, že vlivem chyby v aplikaci může dojít
k nemalým materiálním škodám na kalibrovaném modelu (ceny vybraných
malosériových modelů se pohybují až v~desítkách tisíc Kč).

Dalším přirozeným požadavkem bylo, aby programovací jazyk měl rozumnou podporu
pro komunikaci se sériovým portem, neboť toto rozhraní bude použito pro
komunikaci s měřicím vozem a digitální centrálou DCC.

Dalším kritériem při výběru programovacího jazyka byla jeho předchozí znalost
autorem práce. Zejména u kritických aplikací je totiž nanejvýš důležité, aby
programátor vědět, co dělá, když píše kód.

Při rozmýšlení nad technologií aplikace bylo rozhodnuto, že program by měl
mít grafické uživatelské rozhraní, neboť je nanejvýš vhodné zobrazovat přehledně
aktuální průběh kalibrace a neboť jsou běžní uživatelé na GUI aplikace zvyklí.

Zásadní je pak otázka typu aplikace. Autor uvažovat aplikace desktopové,
mobilní, webové, dokonce i embedded! U embedded aplikací je nevýhoda, že
obsluha s aplikací typicky nemůže příliš interagovat, u mobilních aplikací
je v kontextu této práce omezující zejména zobrazovací plocha zařízení, na
kterou se jednoduše nevejde tolik dat, kolik by aplikace chtěla zobrazovat.
Navíc vývoj pro jiné zařízení, než na kterém kód píšete, vždy přináší nějakou
režii navíc.

Výhodou webové aplikace by byla snadná přenositelnost na jiné počítače,
aplikace by dokonce mohla být přístupná i z mobilního zařízení. Bohužel, proti
tomuto formátu aplikace hovoří fakt, že autor práce nemá s webovými aplikacemi
dostatečné zkušenosti a už vůbec si netroufá v nich psát kritický kód.
Oproti desktopovým aplikacím jsou interaktivní webové aplikace navíc o něco
složitější (zejména co se týče předávání signálů z GUI).

Další možností pak byly desktopové aplikace. Vzhledem k požadavkům uvedeným
výše a k faktu, že se jedná už o trochu rozsáhlejší projekt, jako nejvhodnější
se autorovi jevil jazyk \texttt{C++} s použitím frameworku \texttt{Qt}.
Argumentem pro tento framework byly dobré osobní reference, které na
\texttt{Qt} autor práce dostal os kolegů. Ukázalo se také, že Qt poskytují
nejde výbornou multiplatformní podporu pro GUI, ale také stejnou
multiplatformní podporu pro sériový port. Bylo tedy rozhodnuto, že výsledná
aplikace bude psána v programovacím jazyce \texttt{C++} (konkrétně
\texttt{C++14}) za použití frameworku \texttt{Qt} a bude multiplatformní
(minimálně mezi OS \texttt{Window} a \texttt{Linux}). Důvodem pro
multiplatformnost je především autorova záliba v Linuxu a touha udělat aplikaci
dostupnou i pro běžné smrtelníky.

\section{WSM Speed Reader}

První aplikací je \textit{WSM Speed Reader} -- jednoduchý zobrazovač vyčtených
dat. Aplikace se skládá z~jednoho okna a je především typu
\textit{proof-of-concept}. GUI aplikace je zachyceno na obrázku
\ref{wsm-speed-reader-gui}.

TODO figure: wsm-speed-reader-gui

Aplikace se připojuje k sériovému portu, ze zadaných parametrů vypočítává
rychlost a ujetou vzdálenost vozidla a tyto údaje zobrazuje. Umožňuje ukládání
vyčtených dat do souboru (především pro vývoj) a ukládání nastavení aplikace
do konfiguračního souboru. Aplikace je volně dostupná pod licencí Apache
License v2.0\ref{wsm-speed-reader}.

\section{Automatic Calibration}
