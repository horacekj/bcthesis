There are many interesting challenges coming with the computer-based model
railroad control systems. The problem of a synchronization of a speed and a
breaking curve of a model railroad vehicles is a one of these challenges.
This thesis implements one of the possible approaches to the problem: it
presents a brand-new own detector for a model railroad speed measurement -- so
called measure car. The measurement is based on an optosensor. Bluetooth is
used for data transmission. Measured values are analyzed and an algorithm for
suppressing the errors is developed. This thesis later develops a software for
modification of the parameters of a locomotive decoder so a user-defined speed
is assigned to a specific speed step. The software was designed as a desktop
windows application in a \texttt{Qt} framework with emphasis on a high
usability. The developed technologies allowed the calibration process to be
significantly more accurate and faster. This contributed to the more effective
and collisionless flow of the model railroads.
