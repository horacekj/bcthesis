There are many interesting challenges coming up with the computer-based model
railroad control systems. The problem of speed and breaking curve
synchronization of a model railroad vehicles is one of these challenges. This
thesis implements one of the possible approaches to the problem: it presents a
novel self-made detector for a model railroad speed measurement – a so called
measurement car. The measurement is based on an optosensor. Bluetooth is used
for data transmission. The measured values are analyzed and an algorithm for
suppressing the errors is developed. The author developes a software for
modification of the parameters of a locomotive decoder. The aim of the
application is to assign a user-defined speed to a specific speed step. The
software is designed as a desktop windows application in a \texttt{Qt}
framework with an emphasis on high usability. The developed technologies
allowed the calibration process to be significantly faster and more accurate.
This contributed to more effective and collisionless flow of the model
railroads.
