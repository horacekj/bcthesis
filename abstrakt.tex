S~příchodem moderních počítačově řízených modelových kolejišť vyvstávají mnohé
problémy spojené s~automatizací provozu. Jedním z~těchto problémů je
synchronizace rychlostí a~brzdných křivek digitálně řízených hnacích vozidel.
Tato práce implementuje jeden z~možných přístupů k~řešení tohoto problému:
navrhuje vlastní detektor pro měření rychlosti modelu -- tzv. měricí vůz.
Měření rychlosti je založeno na optickém senzoru, komunikace s~počítačem
probíhá pomocí protokolu Bluetooth. Naměřená data jsou analyzována a~následně
je navrhnut způsob postprocessingu ostraňující z~těchto dat chyby.
Nad technologií vozu tato práce vytváří software, který na základě změřené
rychlosti v~reálném čase upravuje parametry hnacího vozidla tak, aby bylo
dosaženo synchronizace rychlosti. Software byl navržen jako desktopová okenní
aplikace ve frameworku \texttt{Qt} s~důrazem na robustnost a~snadnou
použitelnost uživatelem. Vyvinuté technologie výrazně zpřesnily a~zkrátily
proces kalibrace.
