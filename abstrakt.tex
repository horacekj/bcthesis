S~příchodem moderních počítačově řízených modelových kolejišť vyvstávají mnohé
problémy spojené s~automatizací provozu. Jedním z~těchto problémů je
synchronizace rychlostí a~brzdných křivek digitálně řízených hnacích vozidel.
Tato práce implementuje jeden z~možných přístupů k~řešení tohoto problému:
navrhuje vlastní detektor pro měření rychlosti modelu -- tzv. měricí vůz a~nad
tímto vozem staví software, který na základě změřené rychlosti v~reálném čase
upravuje parametry hnacího vozidla tak, aby bylo dosaženo synchronizace
rychlosti. Měřicí vůz je založen na optickém senzoru a~bezdrátové komunikaci
s~počítačem pomocí protoolu Bluetooth. Ukázalo se, že naměřená data vykazují
periodické chyby. Tyto chyby tedy byly eliminovány. Software byl navržen jako
desktopová okenní aplikace ve frameworku \texttt{Qt} s~důrazem na robustnost
a~snadnou použitelnost uživatelem. Vyvinuté technologie výrazně zpřesnily
a~zkrátily proces kalibrace.
